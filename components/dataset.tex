\subsection{Standard format dataset packages}
\label{sec:comp:dataset}

Derived from \S\ref{sec:design:test}.

The DM subsystem curates a number of ``datasets'': collections of data related by some underlying theme or use case.
These themes might include originating from the same instrument or facility (e.g. testdata\_cfht); being used to test a particular package (e.g. testdata\_jointcal); or addressing some particular science case (e.g. ap\_verify\_hits2015).

Currently, DM's datasets are heterogeneous: there is no accepted standard for the type of data that a dataset should contain, and nor is there any standard for where the dataset is stored or how it is curated.
Some of DM's datsets are stored on \gls{gpfs} at the LSST Data Facility; some are made available through Git LFS\footnote{\url{https://git-lfs.github.com}; \url{https://developer.lsst.io/git/git-lfs.html}}; some contain only raw data; some contain calibration data; some contain processed data; some are regularly updated; some have documentation describing in detail what the dataset contains.

This lack of standardization limits the reuse of datasets (it is hard to reuse a dataset curated for one purpose for some other test unless one fully understands its contents) and means that developers often struggle to find the most appropriate data to test or debug some particular algorithm.

\begin{recommendation}
    {rec:comp:dataset:standard}
    {A standardized format for dataset repositories should be adopted across DM}
Obviously, not all repositories will have exactly the same contents: in some cases, it may be necessary for a repository to contain (say) calibration products, while in others they may be inappropriate.
However, it should be possible to establish at a glance what the contents of each dataset is; if calibration products \emph{are} included, it should be immediately obvious what they are and how to apply them.
\end{recommendation}

The key desiderata for standardizing the format of test datasets are:

\begin{itemize}
\item{
    Developers would like low-friction access to test datasets.
    A central location where developers can look up what has been curated is desired.
}
\item{
    Developers would like datasets at multiple scales and representative of various data quality and observing conditions.
    The properties of these datasets must be well understood.
}
\item{
    Besides raw (unprocessed) data, intermediate and final data products from various stages of pipeline processing are desired.
    They facilitiate testing of algorithms which are only relevant to later parts of the pipeline or analysis codes without the need of regenerating the products.
}
\item{
    Datasets require continuing maintenance.
    Data products based on a recent software release are usually wanted.
}
\end{itemize}

These considerations aside, this WG does not make a specific recommendation about the detailed layout of datasets, beyond noting the existence of the Common Dataset Organization and Policy \footnote{\url{https://developer.lsst.io/services/datasets.html}} which might form a convenient basis for further development.

Regardless of the layout adopted, expect that the dataset will need to evolve with time: as new versions of the LSST code are released, expectations about the both the contents of data repositories and the format of persisted data products will change, occasionally in backwards-incompatible ways.
For a dataset to remain useful, active curation is required.

\begin{recommendation}
    {rec:comp:dataset:owner}
    {Each dataset should have an explicitly named product owner}
Product owners are responsible for ensuring that the content and use cases of the datasets are well described and compatible with recent stack versions.
The owner of a dataset could often be the a member of team with immediate use cases and knowledge of the relevant camera package.
\end{recommendation}

The variety of different sources and use-cases for datasets means that they span a wide range of sizes.
It is therefore impractical to store and distribute them in the same way.
Instead, the QAWG suggests they can usefully be divided into two categories: \emph{small} and \emph{large}.

\begin{description}

\item[Small] datasets are those smaller than 100\,GB in total size.
They are intended for use, for example:
\begin{itemize_single}
    \item{as input test data in \gls{ci} jobs;}
    \item{as example data in documentation, demos, and tutorials.}
\end{itemize_single}
We recommend that they are:
\begin{itemize_single}
    \item{packaged as EUPS products;}
    \item{made publicly available on GitHub;}
    \item{stored as Git LFS repositories as needed;}
    \item{versioned with DM software releases.}
\end{itemize_single}

\item[Large] datasets are bigger than 100\,GB in total size.
They are intended for use, for example:
\begin{itemize_single}
    \item{in large scale integration tests;}
    \item{in flushing out edge cases which may not be apparent from smaller data volumes;}
    \item{for archiving outputs from engineering facilities like the Camera test stands.}
\end{itemize_single}
We recommend that they are:
\begin{itemize_single}
    \item{made available on project-provided shared network filesystems (i.e. \gls{gpfs});}
    \item{protected under a disaster recovery policy.}
\end{itemize_single}

\end{description}

\begin{recommendation}
    {rec:comp:dataset:storage}
    {Datasets may be stored on either shared filesystems or Git LFS as appropriate, depending on the total size of the dataset}
\end{recommendation}
