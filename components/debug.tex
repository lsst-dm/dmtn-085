\subsection{Updated pipeline debugging system}
\label{sec:comp:debug}

This component is derived from \S\ref{sec:design:debug}.

The existing pipeline debugging system, \texttt{lsstDebug}, provides useful
capabilities, but --- perhaps due to an idiosyncratic user interface and lack
of documentation --- is frequently not properly exploited. Since the total
amount of code involved in the old system, both implementing it and using it,
is modest, we suggest its wholesale replacement by an alternative, more
approachable, system.

\begin{recommendation}
    {rec:comp:debug}
    {Develop a new pipeline instrumentation and debugging system, replacing lsstDebug}
The total effort expended here should be modest: we expect that design,
implementation and documentation of a new system should take no more than one
full time equivalent month. Converting existing code make take somewhat
longer.
\end{recommendation}

The capabilities of the new system would remain broadly the same as lsstDebug.
Specifically, enabling debugging mode would set a flag within the codebase.
Developers can check for that flag, and take appropriate actions (e.g.
spawning additional plots) if it is set. However, establishing a baseline
expectation of what actions are appropriate or expecting within this debugging
context will require input from the leadership team.

\begin{recommendation}
    {rec:comp:debug:docs}
    {Guidelines for the effective use of the pipeline debugging system should be supplied to developers}
These guidelines should be supplied by the Science Pipelines Product Owners
and the LSST Software Architect, and made available through the DM Developer
Guide \citeds{DevGuide}.
\end{recommendation}

Developers and users rightly expect that the debugging system will provide an
idiomatic user interface. We suggest that the complexity of the existing
lsstDebug interface has been a bar to adoption. We therefore propose that
simplicity be key aim of the new system. In that vein, we suggest that the
debugging system should be controlled by a single Boolean parameter, with no
further user-driven customization, and with very explicit documentation. We
acknowledge that this is a direct trade-off between granularity and simplicity,
and suggest that history drives us to err on the side of simplicity.

\begin{recommendation}
    {rec:comp:debug:use}
    {Debugging mode should be binary: it is either enabled or disabled, with no further configuration}
    For example, within a \texttt{Task}, one might check if debugging mode is
    enabled with a statement like \texttt{if self.config.debug: ...}.
\end{recommendation}

The natural consequence of this is that debugging of tasks involved by some
form of middleware (e.g. \texttt{CmdLineTask}, \texttt{PipelineTask} called by
an executor) should receive an explicit argument enabling debugging. All other
callables which support debugging must take an explicit argument which enables
it.
