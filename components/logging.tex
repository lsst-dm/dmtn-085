\subsection{Logging}

\assign{Simon}

Logging is an important aspect of running large data processing.
It is also integral to quality assessment as the logging information provides important contextual information when inspecting data for quality issues.
Specifically, log messages presenting information about the processing: e.g. PSF width, number of stars used in a model fit, can indicate problems with the algorithmic behavior or input data.
Logging also provides information about potential causes for unexpected termination including exit codes and exeptions.

While logs can be straightforward to parse when only a small number of concurrent processes are being used, they quickly become harder to understand as the number of processes increases.

The logging component has the following attributes:
\begin{itemize}
\item Configurable for logging granularity. For example, \texttt{INFO}, \texttt{WARN}, \texttt{ERROR}.
\item Trivial to retrieve time ordered logs for a thread.
\item Possible to retrieve logs, exit status, or exceptions for threads ending in an unexpected state.
\item Possible to retrieve enough information to rerun the data units that were unprocessed because of processes ending in unexpected state.  KSK: perhaps this is solved by the provenance system.
\item Searchable per process for regexp and timestamp.
\end{itemize}

Derived from \S\ref{sec:design:debug}.
