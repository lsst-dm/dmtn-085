\subsection{CI system updates}
\label{sec:comp:ci}

Derived from \S\ref{sec:design:test}.

DM's ``Jenkins'' \gls{ci} system provides a number of essential services to the Project and is a core part of developer workflow.
However, the QAWG suggests that a relatively modest set of changes could dramatically improve its usefulness to developers and to the project overall.

\subsubsection{Notifications and failures}

Most importantly, the QAWG suggests that Jenkins should be more helpful in enabling rapid response to build failures.

\begin{recommendation}
    {rec:comp:ci:notify_stakeholders}
    {When running regularly scheduled (timer) jobs on the \texttt{master} branch of any releasable product, any build failure should be announced prominently to key stakeholders}
Those stakeholders should include:
\begin{itemize}
  \item{Senior DM management (DM Project Manager, DM Software Architect, Science Pipelines T/CAMs and Science Leads);}
  \item{The developer who caused the failure, if it is possible to identify them}
\end{itemize}
The term ``prominent'' should be taken to indicate a personalised message (e.g. e-mail, direct Slack message), not a general posting in a Slack channel which regularly sees traffic.
\end{recommendation}

\begin{recommendation}
    {rec:comp:ci:integration_failure}
    {The Developer Guide should provide guidance about expected responses to Jenkins failures}
For example, the policy for handling integration test (e.g. ci\_hsc) failures a merge must be documented: who is responsible for checking for failure? Should the merge be reverted? Who is responsible for doing so?
\end{recommendation}

\subsubsection{Execution environment}

DM requires a range of external packages (NumPy, SciPy, matplotlib, etc).
Versioning constraints on these are checked by ``stub'' packages at configuration time.
Typically, the only check applied is that the available version exceeds some minimum: no maximum versioning constraint is enforced.
This exposes us to possible API breakage in newer versions of upstream packages.
We do not regard this as requiring urgent intervention, but the team may wish to consider enforcing version maximums at some point in the future.

For maximum coverage, all tests should be run with all possible combinations of external packages.
In practice, of course, this is prohibitive.
However, it is essential that the code be shown to work with the minimum versions of all packages documented as required.

\begin{recommendation}
    {rec:comp:ci:environment}
    {The versions of external packages used in the Jenkins system must always correspond to the minimum versions specified in stub packages and/or in the document list of prerequisites}
\end{recommendation}

We note that the details of this recommendation may be subject to alteration, depending on changes to the DM release process and the way it interacts with third party dependencies.
However, the spirit of the recommendation --- that our code must be checked with the library versions it claims to support --- should stand in any eventuality.

\begin{recommendation}
    {rec:comp:ci:versions}
    {The project should adopt a single source of dependency information and versions}
This might consist of ``stub'' EUPS packages, or of a Conda environment, or of a list of packages on a website, but there must be \emph{one} unambiguously authoritative source of information.
\end{recommendation}
