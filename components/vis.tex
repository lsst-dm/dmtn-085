\subsection{Visualization}
\label{sec:comp:vis}

Derived from \S\ref{sec:design:debug}.

Broadly, we regard ``visualization'' as an umbrella term covering both visualizations derived from catalogs as well as image display and manipulation.
The concerns expressed by developers and others are, on the whole, common to both; the solutions may not be.

For both visualization regimes, the predominant request from developers is that the project provide them with clear guidance as to both what resources will be provided and supported by the LSST construction effort (for example, tools like Firefly or abstractions like \texttt{afwDisplay}) and which tools they are required or expected to use in the interest of maintaining a coherent and consistent codebase and set of outputs.

In this section, we concentrate on ad-hoc visualization in support of the regular pipeline developer.
In \S\S\ref{sec:design:drill} and \ref{sec:comp:drill}, we describe the design of an interactive ``drill-down'' tool, which will provide a number of plotting and data exploration capabilities.
It is our expectation that, however comprehensive such a tool might become, there will always be a necessity for individual developers to be able to quickly investigate their data in as flexible a way as possible; conversely, wherever practical we should enable developers to exploit, and encourage them to remain consistent with, the capabilities and conventions delivered by the drill-down tool.
These sections should therefore be regarded as complementary.

\subsubsection{Catalog visualization}
\label{sec:comp:vis:catalog}

The group notes that there are many contexts in which visualizations derived from catalogs might be required (for example, in-line display in a Jupyter notebook, persisting plots from a debugging session, as described in \S\ref{sec:comp:debug}, preparing plots for publication, etc), which may all have substantially different requirements.
We also note that there exists a diverse infrastructure of scientific plotting and data exploration tools in the Python community, a comprehensive selection of which has been collated by the PyViz project\footnote{\url{http://pyviz.org}}, which also provides documentation on effectively using them in conjunction with each other.
Given that, we regard it as unnecessary, and indeed undesirable, for LSST to attempt to standardize on any particular plotting tool: we should rather encourage developers to exploit community resources with minimal overhead.

\begin{recommendation}
  {rec:comp:vis:catalog:pyviz}
  {DM should formally adopt the PyViz ecosystem}
This adoption would include, for example, including PyViz tools in a regular installation of the LSST Stack; providing training and documentation for developers and --- crucially --- developing interfaces which enable LSST conventions (afw tables, the Data Butler) to be used in the PyViz context.
\end{recommendation}

Many visualization use cases involve manipulating data at a larger scale than can conveniently be done on a single compute node.
Within the PyViz ecosystem, Dask\footnote{\url{http://dask.pydata.org/en/latest/}} is the preferred approach.
We note that there many be some redundancy between Dask and the LSST middleware (Butler, PipelineTask and appropriate executors, etc).
However, Dask provides a convenient, easy to install and use solution which can immediately address developer needs.

\begin{recommendation}
  {rec:comp:vis:catalog:dask}
  {DM should adopt Dask to enable users to work with larger than memory data}
This might be achieved by providing users with the ability to spin up Dask clusters on demand using (say) Kubernetes, or by providing a Dask cluster at the LSST Data Facility to which users can connect.
If ongoing middleware development renders this obsolete, then it can be retired.
\end{recommendation}

\subsubsection{Image visualization}
\label{sec:comp:vis:image}

The concern among developers about what they ``ought'' to be doing is particularly acute when considering image visualization: developers are universally familiar with SAOImage DS9\footnote{\url{http://ds9.si.edu/}}, but also aware of other \texttt{afwDisplay} back-ends (Ginga\footnote{\url{https://ginga.readthedocs.io/}}, Matplotlib\footnote{\url{https://matplotlib.org}}), and aware that project resources are being spent on Firefly.
This uncertainty is compounded because most developers are unaware of long-term plans for Firefly, and even among system management there is uncertainty about the Firefly development timeline and the extent to which Firefly development is able to accommodate emergent work and requests in response to ongoing development.

\begin{recommendation}
  {rec:comp:vis:image:guidance}
  {DM should provide clear, written guidance to developers about the availability, status and expected usage of image display tools}
\end{recommendation}

We identify two separate regimes of image visualization:

\begin{enumerate}

  \item{A laptop with a small amount of data, where images and other results are inspected with local tools;}
  \item{A server-based environment, like the LSST Science Platform's Notebook Aspect, where centrally provided services have access to large amounts of data.}

\end{enumerate}

In the first case, a desktop application like DS9 may be convenient.
However, in the second regime, a server-side visualization tool like Firefly provides more convenient access to the data: it enables us to bring image display tools to the data, rather than image data to the display.

In addition to the above, we identify three separate ways in which users may wish to use a visualization tool:

\begin{enumerate}

  \item{As a standalone tool (albeit controllable via a Python API);}
  \item{Embedded within a Jupyter Notebook;}
  \item{As a separate ``tab'' within the JupyterLab environment.}

\end{enumerate}

We note that there are a number of LSST-specific requirements on image visualization.
These include display of LSST-style image masks, \texttt{Footprint}s, and other specialized classes, as well as a mechanism for conveniently visualizing the full focal plane.

Previous drafts of this document identified Firefly as the only viable choice for addressing LSST-specific requirements and operating in all of the environments required by LSST.
Unfortunately, following the recent changes to the scope of the DM Science User Interface and Tools group (\citeds{DMTN-096}) it no longer seems practical to go ``all in'' on Firefly: we do not expect Firefly to receive the concentrated attention on usability, responsiveness and LSST specific features which it would require to become a subsystem-wide standard.

With some reluctance, therefore, the working group concludes that for the remainder of construction DM developers will continue to operate with a heterogeneous mixture of visualization tools.
We regard this as a significant threat to the overall efficiency of the construction effort, but are unable to suggest an alternative approach within the established schedule, scope and budgetary constraints on the subsystem.
We are further concerned about the lack of high-quality visualization tooling available to the commissioning and science validation efforts --- and, indeed, about the knock-on effects on what tooling will be available to analyse LSST data during the operational era.
However, these concerns are out of scope for this group.
