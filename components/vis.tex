\subsection{Visualization}
\label{sec:comp:vis}

Derived from \S\ref{sec:design:debug}.

Broadly, we regard ``visualization'' as an umbrella term covering both visualizations derived from catalogs as well as image display and manipulation.
The concerns expressed by developers and others are, on the whole, common to both; the solutions may not be.

For both visualization regimes, the predominant request from developers is that the project provide them with clear guidance as to both what resources will be provided and supported by the LSST construction effort (for example, tools like Firefly or abstractions like \texttt{afwDisplay}) and which tools they are required or expected to use in the interest of maintaining a coherent and consistent codebase and set of outputs.

In this section, we concentrate on ad-hoc visualization in support of the regular pipeline developer.
In \S\S\ref{sec:design:drill} and \ref{sec:comp:drill}, we describe the design of an interactive ``drill-down'' tool, which will provide a number of plotting and data exploration capabilities.
It is our expectation that, however comprehensive such a tool might become, there will always be a necessity for individual developers to be able to quickly investigate their data in as flexible a way as possible; conversely, wherever practical we should enable developers to exploit, and encourage them to remain consistent with, the capabilities and conventions delivered by the drill-down tool.
These sections should therefore be regarded as complementary.

\subsubsection{Catalog visualization}
\label{sec:comp:vis:catalog}

The group notes that there are many contexts in which visualizations derived from catalogs might be required (for example, in-line display in a Jupyter notebook, persisting plots from a debugging session, as described in \S\ref{sec:comp:debug}, preparing plots for publication, etc), which may all have substantially different requirements.
We also note that there exists a diverse infrastructure of scientific plotting and data exploration tools in the Python community, a comprehensive selection of which has been collated by the PyViz project\footnote{\url{http://pyviz.org}}, which also provides documentation on effectively using them in conjunction with each other.
Given that, we regard it as unnecessary, and indeed undesirable, for LSST to attempt to standardize on any particular plotting tool: we should rather encourage developers to exploit community resources with minimal overhead.

\begin{recommendation}
  {rec:comp:vis:catalog:pyviz}
  {DM should formally adopt the PyViz ecosystem}
This adoption would include, for example, including PyViz tools in a regular installation of the LSST Stack; providing training and documentation for developers and --- crucially --- developing interfaces which enable LSST conventions (afw tables, the Data Butler) to be used in the PyViz context.
\end{recommendation}

Many visualization use cases involve manipulating data at a larger scale than can conveniently be done on a single compute node.
Within the PyViz ecosystem, Dask\footnote{http://dask.pydata.org/en/latest/} is the preferred approach.
We note that there many be some redundancy between Dask and the LSST middleware (Butler, PipelineTask and appropriate executors, etc).
However, Dask provides a convenient, easy to install and use solution which can immediately address developer needs.

\begin{recommendation}
  {rec:comp:vis:catalog:dask}
  {DM should adopt Dask to enable users to work with larger than memory data}
This might be achieved by providing users with the ability to spin up Dask clusters on demand using (say) Kubernetes, or by providing a Dask cluster at the LSST Data Facility to which users can connect.
If ongoing middleware development renders this obsolete, then it can be retired.
\end{recommendation}

\subsubsection{Image visualization}
\label{sec:comp:vis:image}

There are two approaches to doing image visualization.
When a developer is working on a laptop with a small amount of data and viewing image results locally then DS9 or an equivalent desktop application seems to be the best solution, primarily because of the ease of setup.

In the server-based environment of the LSP Notebook Aspect, where services can be more centrally provided, the use of Firefly-based image visualization becomes convenient and will permit providing more advanced and LSST-customized behavior.
When the data themselves are already being manipulated on a remote server, the use of a co-located visualization server can provide better performance.

Experience so far has shown that different users, at different times, prefer one or the other of the models of having image displays ``next to'' their Python session --- the only approach available with DS9 --- or embedded in a scrolling notebook.
Which is more useful scientifically or pedagogically often depends on what the immediate purpose is.
In the latest JupyterLab environment, a third option is available of having image visualization in a separate ``tab'' in the JupyterLab layout.
Each of these modes is expected to be possible with Firefly; the first (visualization ``next to'' Python) is routine, the scrolling widget mode has been demonstrated, and the JupyterLab extension development needed for the third is scheduled.

To be effective for QA applications Firefly should have the following functionality:

\begin{itemize}
  \item{Work as an implementation of the \verb|afw.display| API;}
  \item{Provide a low-level Python API that allows its use with data that may not be in \verb|afw| formats; this API may also allow Firefly to}
  \item{Provide additional or more advanced features than are supported in the \verb|afw.display| base class\footnote{These might get migrated to inclusion in \texttt{afw.display} over time.};}
  \item{Work as an extension in Jupyter Lab (or example: click on a FITS file and see it show in Firefly);}
  \item{Work as a Jupyter Notebook widget, supporting JupyterLab's ``tear-off'' feature to move it to a tab;}
  \item{Provide a way to overlay tabular data on images;}
  \item{Provide HiPS viewing;}
  \item{Provide visualizations for LSST's image masks, object \verb|Footprints|, and other specialized classes;}
  \item{Support all major QA use cases and ensure good performance;}
  \item{Implement full focal plane visualization; and}
  \item{Work as a standalone tool, supporting exploration of generic and LSST-created image files directly from the tool, without the need for a Python driver.}
\end{itemize}

The SUIT group is working with the other LSP groups to ensure that Firefly services are provided to all LSP instances, both the ``planned'' centrally provided instances such as the PDAC and \verb|lsst-lspdev|, and the ``pop-up'' instances that are used for supporting workshops and tutorials.
The group will also continue to provide a service for the legacy Unix-login development environment at NCSA.
In addition, the group plans to provide Docker-based deployment images for different user-driven installation scenarios, ranging from a laptop-based Firefly server that provides single-user visualization support without the need for large memory or computational requirements, to a central-site-oriented server that provides high performance by taking advantage of a large number of cores and substantial memory.
