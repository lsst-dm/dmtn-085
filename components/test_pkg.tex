\subsection{Standard format test packages}
\label{sec:comp:test_pkg}

Derived from \S\ref{sec:design:test}.

Analogously to various inconsistent forms of packaging datasets described in \S\ref{sec:comp:dataset}, we also have a variety of different styles of packages which are fundamentally designed to perform some form of integration testing: that is, executing a test suite which exercises multiple constituent parts of the DM simultaneously.
Broadly, we consider that these fall into one of two categories:

\begin{enumerate}
    \item{Scons-based execution, including ci\_hsc and ci\_ctio0m9; or}
    \item{Shell-script based, including validate\_drp and ap\_verify\footnote{But note that this shell-script entry point may cover a variety of lower-level implementation details, such as LSST-standard \texttt{CmdLineTask}s, Python scripts, or further shell scripting.}.}
\end{enumerate}

Often packages of this type are designed for automatic execution within the Jenkins environment, typically on nightly timers, but we note that:

\begin{itemize_single}
    \item{Even for packages designed with Jenkins in mind, being able to run standalone is still generally essential;}
    \item{We expect future integration tests to involve executing large-scale jobs which will run on cluster-scale hardware at the Data Facility. Such jobs may be coordinated by Jenkins, but will (likely) not run entirely on Jenkins build agents.}
\end{itemize_single}

We consider that broadly the same benefits to adopting a consistent structure in these packages apply here as in \S\ref{sec:comp:dataset}.
In particular, adopting a standardized design would enable:

\begin{itemize}
    \item{Developers to make use of existing tests when developing new algorithms with minimal overhead;}
    \item{New tests to be added without significant design work.}
\end{itemize}

\begin{recommendation}
  {rec:comp:test_pkg:standard}
  {A standardized test package design should be developed which addresses all existing use cases}
Existing test packages (lsst\_dm\_stack\_demo, ci\_hsc, validate\_drp, ap\_verify, etc) should be adapted to the new design, and, where possible, merged with each other.
\end{recommendation}

Finally, we suggest that the existing variety of test packages reflects a lack of clarity and orthogonalization as to exactly what functionality can and should be tested in \gls{ci} and on what cadences.

\begin{recommendation}
  {rec:comp:test_pkg:ci_doc}
  {A coherent plan for integration testing at all scales should be developed and published}
Such a plan should then drive the development of the test package standard discussed above.
Note \jira{DM-15348} in this context.
\end{recommendation}
