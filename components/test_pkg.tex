\subsection{Standard test package design}
\label{sec:comp:test_pkg}

Derived from \S\ref{sec:design:test}.

\assign{Hsin-Fang}

Currently, automatic continuous integration tests are performed via
multipla packages under two designs: (1) Scons-based execution, including
ci\_hsc and ci\_ctio0m9, and (2) exeuction through shell scripts
in validate\_drp.  Both ci\_hsc and validate\_drp are run in
Jenkins and triggered by timers every night (\S\ref{sec:comp:ci}).

It's QAWG's understanding that the validate\_drp scripts will
eventually replace ci\_hsc and ci\_ctio0m9, and a set of test scripts
will be run in a meta-package named lsst\_ci.  However, at the time
of writing, the validate\_drp scripts test only the single frame
processing step, while ci\_hsc exercises almost the entire end-to-end
DRP pipelines.  There has not been sufficient resources in implementing
further processing in validate\_drp.  Similarly, the lsst\_dm\_stack\_demo
repository should be converted into an EUPS product and a test
script added to lsst\_ci for execution (\jira{DM-14806}).

The QAWG recommends priority to unify the CI test package design and
finish the transition to validate\_drp. If such effort cannot
be allocated, documentations should be added to clearly describe
the status quo, and recommendations for developers during the
transition should be added to the Developer Guide (similar to
\S\ref{sec:comp:doc} and \S\ref{sec:comp:ci}). Before validate\_drp
can replace ci\_hsc and ci\_ctio0m9, the packages should be maintained.


\begin{recommendation}
  {rec:comp:provenance}
  {The CI design should address the use cases of existing packages
   (lsst\_dm\_stack\_demo, ci\_hsc, validate\_drp) and unify them into
   one consistent design}
\end{recommendation}
\begin{recommendation}
  {rec:comp:provenance}
  {The current status quo of the CI system should be well documented
   and updated during the transition}
\end{recommendation}
