\subsection{Provenance}
\label{sec:comp:provenance}

\assign{Hsin-Fang}

The QAWG recognizes the importance of provenance and the
implementation of the provenance system will impact QA work
significantly.  We recommend high priority to finalize the design
and implementation of the provenance system.

\begin{recommendation}
  {rec:comp:provenance}
  {The design and implementation of the provenance system should have
  high priority in the project scheduling.}
\end{recommendation}

The QAWG believes that the requirements on provenance tracking are
adequate as described in the Data Management Middleware Requirements
(\citeds{LDM-556}) and Data Management Data Backbone Services
Requirements (\citeds{LDM-635}). QA use cases provide no further
requirements.

With the current design of the Gen 3 Middleware, each dataset
can be linked to provenance information such as input datasets,
pipeline definition, configurations, and software version
(e.g. DMS-MWST-REQ-0024 and DMS-MWBT-REQ-0096 in LDM-556).
\gls{QC} metric values will be Butler datasets and have Butler data IDs
(as recommended in \S \ref{rec:metric_dataId}), so
their provenance can be traced like
other datasets. In the Gen 3 Middleware design, Butler/SuperTask
framework is responsible of producing provenance.  For production
runs, the Data Backbone Services may store additional provenance,
for example from the Batch Production Services, besides the
Butler/SuperTask generated provenance.  In the QA use cases, the
primary source of provenance will be the Gen3 Butler/SuperTask
generated provenance.

Regarding provenance of the database records, the QAWG does not
place a strong requirement on per-source provenance.  The current
design is that each database record in the production database is
either ingested from a file, of which the full provenance is traced,
or from an uniquely identifiable execution.  This design does not
directly provide detailed per-source provenance, such as what exact
input images acually contribute to the measurement of a particular
source, but the full inputs that can contribute to the source.  The
provenance tracking of synthetic sources is also unclear.  The QAWG
thinks most low-level information can be uncovered through the drill
down system (\S\ref{sec:design:drill}).  Diagnostic information can
also be computed in the pipeline codes and stored as additional
columns.  We recommend the tooling to investigate the composition
of coadds be made in the drill down system and does not put this
requirement in the provenance system.

The QAWG notes that some high level aggregate data products that
are derived from provenance data can be useful in QA work.  For
example, the number of images that contribute to each coadd patch
can be obtained from provenance data.  However, the QAWG are not
immediately convinced that it's worth spending significant time
investigating what aggregate products need to be considered.

Derived from \S\ref{sec:design:debug}.

This section should note:

\begin{itemize}

  \item{That provenance is an immediate issue impacting QA work, so a solution
  is a priority;}

  \item{Some requirements as to the granularity at which provenance tracking
  is necessary for QA.}

\end{itemize}
