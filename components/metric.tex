\subsection{Metric collection \& tracking}
\label{sec:comp:metric}

This section is derived from \S\S\ref{sec:design:drill} and \ref{sec:design:test:metrics}.

We note that metrics describing pipeline execution may be considered in two seaprate --- but related --- contexts.
In some cases, we care simply about the absolute value of some metric value: is the performance ``good enough''?
Does it satisfy a requirement?
In other circumstances, we might wish to keep track of a metric value as a time series: does performance change with time?
Can we identify changesets which have introduced regressions (or improvements!)?

We note further that metrics values might be calculated in a number of different contexts.
For example:

\begin{itemize}

  \item{Some metrics refer to characteristics of pipeline execution, like execution time, which \emph{by definition} must be recorded during execution or they are lost\footnote{Of course, it is not a requirement that they by recorded by a dedicated metric tracking system; one could imagine recording execution time by simply recording a log message, and then later parsing log outputs to retrieve it}.}

  \item{Others may only be calculated from intermediate data products, which would not normally be stored for posterity. These must be calculated before those intermediate products are removed.}

  \item{Finally, some are calculated from final science data products, and hence may be calculated at any time after pipeline execution.}

\end{itemize}

\subsubsection{Defining, calculating and collecting metrics}
\label{sec:comp:metric:collect}

The author of pipeline code can, of course, simply print to screen (or to log) a message containing some quantity that they have calculated on the fly during the execution of their code.
This is low overhead and trivial to implement.
It may be combined with the debugging system (\S\ref{sec:comp:debug}) to provide a rich set of diagnostics when the code is executing.
We suggest that there is no advantage to attempting to force some new framework onto developers operating in this mode.

However, at the level of long-term monitoring of pipeline performance, and especially at the level of requirements verification, we suggest that having a standardized, code-based definition of metrics is essential to enable clear and unambiguous comparison of results.
The SQuaRE team has developed the \texttt{lsst.verify.metrics}\footnote{\url{https://github.com/lsst/verify_metrics}} package to facilitiate the centralised definition of metrics, which we regard as a key step in the right direction.

\begin{recommendation}
    {rec:comp:metric:collect:verifymetrics}
    {Formalise the \texttt{lsst.verify.metrics} package as the source of truth for metric definitions, by e.g. describing it in \citeds{LDM-503} and \citeds{LDM-639}.}
\end{recommendation}

The SQuaRE team has also developed the \texttt{lsst.verify} package which enables the convenient packaging of metric values and submission to SQuaSH, the metric tracking dashboard (\S\ref{sec:comp:metric:dashboard}).

We have some concerns that adoption of this system has been slow.
In part, that might be due to it being developed relatively independently of ongoing work on the Science Pipelines and without substantial external input to or review of the design.
However, we believe that there are a small number of concrete steps which can be taken to address this.

\begin{recommendation}
    {rec:comp:metric:collect:verifydocs}
    {Provide a single, reliable set of documentation describing the metric definition and collection system}
    In particular, mentions of old, obsolete packages\footnote{e.g. \texttt{lsst.validate.base}} should be expunged, and a clear set of introductory documentation should be provided which does not refer to informal technotes describing vaguely-specified design goals (\citeds{SQR-017, SQR-019}).
    In the process of developing this documentation, work closely with a named stakeholder in the Pipelines group to ensure that their needs are being adequately met; some redesign of existing code may be necessary.
\end{recommendation}

\begin{recommendation}
    {rec:comp:metric:collect:verifypipe}
    {Develop clear guidelines for integrating metric collection with pipeline code}
    \citeds{DMTN-057} suggested a number of ways in which this might be done, but indecision has caused paralysis.
    The onus is on the Pipelines group to adopt one of these approaches (or develop an alternative)\footnote{Note \jira{DM-16016} in this context.}.
\end{recommendation}

\begin{recommendation}
    {rec:comp:metric:collect:adopt}
    {Pipelines leadership should start using the metric definition and collection system.}
    As the above recommendations are met, this system will be usable.
    However, driving adoption will require proactive measures from pipelines Product Owners and T/CAMs.
\end{recommendation}

\subsubsection{Metric tracking dashboard}
\label{sec:comp:metric:dashboard}

The \gls{squash} system, \citeds{SQR-009}, has been developed by the SQuaRE team to provide ``\gls{dashboard}'' functionality for metric tracking.
At its core, SQuaSH provides a database to which metric values may be submitted using the \texttt{lsst.verify} (\S\ref{sec:comp:metric:collect}) system, and a web-based service for displaying metric values as a time series.
This enables the user to track the evolution of metric values with time, and relate them directly to changes in code or configuration.
SQuaSH has also been designed to provide some limited ``drill-down'' functionality to explore the way in which high-level metrics have been calculated.

To date, SQuaSH has been used to follow a set of metrics derived from high-level LSST requirements and codified in the validate\_drp package.
It has been designed, though, to enable use by individual developers to track metrics which are of interest only to them, or relevant a particular subset of the codebase on which they are working.

The Working Group feels that the major value in the SQuaSH system is in tracking and responding to regressions in performance (be they scientific or run-time) as the code changes.
In this respect, it is in some ways analagous to the \gls{ci} system (\S\ref{sec:comp:ci}), and benefits from many of the same recommendations.

\begin{recommendation}
    {rec:comp:metric:dashboard:alert}
    {SQuaSH should issue alerts to developers and key stakeholders on regressions in important metric values}
    Key stakeholders should include:
    \begin{itemize}
      \item{Senior DM management (DM Project Manager, DM Subsystem Scientist, Pipelines Scientist, Science Pipelines T/CAMs and Science Leads);}
      \item{The developer who casued the regression, if it is possible to identify them (e.g. through commit logs).}
    \end{itemize}
    This will require careful design, as it may be in tension with the desire to enable developers to define arbitrary metrics for their own use: clearly, key stakeholders will not wish to be informed of every developed-defined metric which suffers a regression.
    We suggest that, for example, a ``subscription list'' for each metric be defined, and the key stakeholders automatically be added to it for all metrics deriving directly from high-level requirements.
\end{recommendation}

As with \texttt{lsst.verify}, we worry that there is confusion about how to distinguish metric values measured on different versions of the codebase, configurations, datasets, etc within the SQuaSH system.
For example, it is possible to define and track a metric on an algorithm implemented by code in the \texttt{lsst.pipe.tasks} package.
But that code may be run in multiple different contexts: as part of alert production, data release production, precovery, etc: how does SQuaSH distinguish between all of these environments?
We believe that SQuaSH is capable of this, but existing ``big picture'' documentation is lacking and hard to follow.

\begin{recommendation}
  {rec:comp:metric:dashboard:docs}
  {Provide a single, reliable source of documentation describing the SQuaSH system and a vision for its use in DM-wide metric tracking}
\end{recommendation}

We note that SQuaSH provides some drill-down capability to explore the source of metric values.
We suggest that this should not be the core business of SQuaSH, and prefer to consider a separate drill-down environment (\S\ref{sec:comp:drill}); further development of these capabilities within squash should not be prioritised.

We recognize that developers may wish to submit results to SQuaSH from a variety of systems.
In particular, it must not be dependent upon a particular expection environment (e.g. Jenkins).

\begin{recommendation}
    {rec:comp:metric:dashboard:agnostic}
    {It must be possible to submit metrics to SQuaSH from arbitrary pipeline execution environments}
\end{recommendation}

When handle large datasets, there is value to tracking metric values computed over subsets of the whole.
For example, it may be more relevant to track how photometric repeatability varies over some patch, rather than over the whole sky.

\begin{recommendation}
    {rec:squash:dataId}
    {SQuaSH should be able to store and display metric values per DataId}
    For example, CCD, visit, patch, tract, filter.
\end{recommendation}
