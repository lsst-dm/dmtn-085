\subsection{Documentation and examples}
\label{sec:comp:doctest}

Derived from \S\ref{sec:design:test}.

Across the codebase there are scripts, example code, and other utilities.
Often these are lurking in \texttt{examples} directories in stack packages.
Occasionally they are exposed as Jupyter notebooks, sometimes living in separate repositories.
These examples are generally hard to discover; often, they are only made available to new team members (or external third parties) by chance conversation.

\begin{recommendation}
    {rec:sec:comp:doctest:index}
    {Provide a central location where examples, scripts and utilities which are not fundamental to pipeline execution are indexed and made discoverable}
See also \jira{DM-15807}.
\end{recommendation}

We observe that many of these examples are old, obsolete and often broken.
We note that users attempting to run these examples will frequently report that they are found to be non-functional.
We further observe that developers are unclear about their obligations for updating these examples when writing new code, an issue which is compounded when the code being changed is in a different package from the affected example.
Finally, we regard broken examples as contributing to an actively hostile user experience: no example is better than a misleading or failing one.

\begin{recommendation}
    {rec:sec:comp:doctest:policy}
    {The Project should adopt a documented (in the Developer Guide) policy on the maintenance of example code}
\end{recommendation}

Ultimately, \emph{all} code --- include examples --- should be tested in the \gls{ci} system: see below.
However, pending a mechanism for this, we suggest that:

\begin{itemize}
    \item{Developers are \textit{not} required to search the codebase for examples which may be affected by changes they are making elsewhere;}
    \item{
        When a broken example is discovered, it may be fixed if the changes required are trivial.
        However, if substantial effort would be needed, the example should simply be removed and an issue filed on Jira to request its reinstatement in future.
    }
\end{itemize}

Ultimately, we suggest that examples should be tightly intyegrated with the overall documentation effort.


\begin{recommendation}
    {rec:sec:comp:doctest:ci}
    {The Project should prioritise the development of a documentation system which makes it convenient to include code examples and which tests those examples as part of a documentation build}
\end{recommendation}

There are various technologies which could be adoped to to address this goal\footnote{For example, \href{https://jupyter.org/}{Jupyter notebooks} or \href{http://www.sphinx-doc.org/en/stable/ext/doctest.html}{Sphinx doctests}.}.
The WG suggests that standardizing upon a single technology is essential, but takes no position as to which technology is most appropriate.

Finally, we note that the the same concerns apply to executable code (in \texttt{bin} directories) which is not regularly tested in CI.
