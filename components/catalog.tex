\subsection{Catalog visualization tools}

\assign{Lauren}

Derived from \S\ref{sec:design:debug}.

It is important for developers to be able to easily, and interactively, visualize large quantities of catalog data.
The \href{http://pyviz.org}{PyViz} ecosystem (\href{http://holoviews.org}{HoloViews}/\href{http://datashader.org}{Datashader}/\href{http://dask.pydata.org}{Dask}/\href{http://pandas.pydata.org}{Pandas}) is designed for exactly this purpose, and so the QAWG recommends the following:
\begin{itemize}
    \item{Including up-to-date PyViz packages in the LSST stack,}
    \item{Developing basic tools using this ecosystem that mesh with existing stack infrastructure (e.g., \texttt{Butler}, \texttt{AfwTable}),}
    \item{Providing training to developers on how to use it,}
    \item{Enabling users to use Dask to work with larger-than-memory data, either with the ability to spin up Dask clusters on demand, or (perhaps preferred) to have a central Dask cluster at the LDF to which users can connect. (HFC: I think more investigations are needed before we ask for Dask.  Some features and use cases overlap between Dask and the Gen 3 Middleware. I don't think we can discard using the Gen 3 Middleware here without investigations and discussions.)}
\end{itemize}
