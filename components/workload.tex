\subsection{Workload management system}
\label{sec:comp:workload}

Derived from \S\ref{sec:design:debug}.

A commonly heard complaint from developers is that the logistics of running code at scale for test purposes is too complex and unreliable.
In particular, developers are concerned that sometimes jobs fail to execute without a reason being clearly stated (the job simply disappears from the Slurm queue, without explanation) or that it can be hard to understand from the logs why a job failed or particular output was generated.

The WG reached the conclusion that a wholesale rethinking of workload management is outside the scope of the group's charge.
Instead, we suggest just three key improvements: to logging (to better identify and diagnose failures), to provenance (to better understand why certain data products have been produced) and to documentation (primarily, to avoid confusion over why certain jobs might vanish without trace). Logging and provenance are addressed in \S\ref{sec:comp:log} and \S\ref{sec:comp:provenance} respectively.

\begin{recommendation}
  {rec:comp:workload}
  {Tutorial and reference documentation for developers attempting to run jobs at scale should be refreshed}
  In particular, revised documentation should focus on identifying and resolving common failure modes, and understanding how best to use existing resources, such as the dashboard at \url{https://monitor-ncsa.lsst.org/}, to rapidly diagnose and escalate issues with underlying infrastructure.
\end{recommendation}
