\subsection{Datasets and test infrastructure}
\label{sec:design:test}

The group regards this part of its charge as qualitatively different from the
other two (\S\S\ref{sec:design:debug} \& \ref{sec:design:drill}): where those
focus on servicing the needs of the individual developer or analyst, this
material describes the capabilities that are required by the overall subsystem
to track its progress and verify its deliverables.

We considered the following three aspects of this part of the charge:

\subsubsection{Small-scale unit and documentation tests}

The current unit test and \gls{ci} system functions well. However, we suggest
a number of improvements. These include clearer instructions to developers on
the expectations for testing (discussed in \S\ref{sec:comp:code_quality}) and
assorted minor upgrades to the \gls{ci} system, primarily to enable more
convenient failure notifications and tighter control of the test environment
(\S\ref{sec:comp:ci}).

The single highest priority we identified was the lack of a current system for
performing CI on example code, documentation and (perhaps) Jupyter notebooks.
This has already rendered many of the examples provided in the current
codebase obsolete --- and because they aren't regularly tested, we have no way
to know what else is broken and will fail or otherwise confuse new (or
existing) users. We regard addressing this as one of the single highest
priorities for the project. It is discussed further in
\S\ref{sec:comp:doctest}.

\subsubsection{Integration tests}
\label{sec:design:test:integration}

DM's integration test needs are currently served by a heterogeneous mix of
approaches: from ``unit'' tests which effectively test the integration of
multiple packages, to explicit integration jobs executed by the CI
system --- of which there are multiple types, and no common
implementation standard\footnote{See
\href{https://github.com/lsst/lsst_dm_stack_demo}{lsst\_dm\_stack\_demo},
\href{https://github.com/lsst/ci_hsc}{ci\_hsc},
\href{https://github.com/lsst/validate_drp}{validate\_drp}, etc.} --- to large
scale data processing exercises undertaken periodically at the
\gls{ldf}\footnote{e.g.
\url{https://confluence.lsstcorp.org/display/DM/Reprocessing+of+the+HSC+RC2+dataset}.}.

We posit that a unified and documented approach to integration testing will
provide lower implementation overheads, fewer surprises for developers, and
more predictable coverage for the project. We expand on the considerations and
design for such a system in \S\ref{sec:comp:test_pkg}.

Developing a standardized approach to integration testing requires a
standardized approach to the management of test datasets: this is addressed in
\S\ref{sec:comp:dataset}. It also impacts upon the \gls{ci} system, and hence
is relevant to \S\ref{sec:comp:ci}.

Finally, we note that the way in which pipelines are defined and executed will
likely evolve rapidly over the remainder of the construction period as
technologies like PipelineTask and Butler Generation 3 come into use. Given
the rapidly-moving nature of this work, the QAWG regards them as out of scope,
except insofar as we urge that QA tasks should be implemented and kept up to
date with these new frameworks as they become available.

\subsubsection{Metric and performance tracking}
\label{sec:design:test:metrics}

DM already has a metric-tracking system based on \gls{squash}. This
effectively tracks the time evolution of a number of \glspl{aggregate metric}
based on data processing under controlled conditions (effectively, the results
of selected integration tests, carried out as per
\S\ref{sec:design:test:integration}). This basic machinery is excellent, but
has suffered from low adoption among DM developers, and it is not clear that
any regular checking of or acting upon trends in the calculated metrics is
being undertaken.

We therefore propose a series of updates and refinements to \gls{squash},
focusing on a series of modest enhancements to its capabilities to alert DM
developers and management to regressions (or improvements) in selected
performance metrics. These are developed in \S\ref{sec:comp:metric:dashboard}.

The primary means by which data is submitted to \gls{squash} is through the
\href{https://github.com/lsst/verify}{\texttt{lsst.verify}} framework and
associated metric definitions. The mechanisms by which this framework is
integrated with the Science Pipelines require further definition (refer to
\citeds{DMTN-057} and \citeds{DMTN-098} for discussion) and it has never
undergone a design review or acceptance process.  Clearer ownership and
direction for this part of the system are essential to drive adoption. We
discuss this in more detail and provide suggestions in
\S\ref{sec:comp:metric:collect}.
