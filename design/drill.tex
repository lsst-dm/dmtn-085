\subsection{Drill down}
\label{sec:design:drill}

Drill-down workflows center on the need to quickly and efficiently identify
data processing problems. Typically, these will be identified from
discrepancies identified at higher levels of summary and aggregation.

We therefore envisage a drill-down system which provides rapid retrieval of
relevant quantities (metric values, image cutouts, catalog overlays, etc.)
combined with readily-(re)configurable interactive plotting tools. This is
provided in a browser-based tool which enables the user to rapidly access
successive layers of detail on aggregated metrics (effectively
``de-aggregating'' them on demand), and ultimately enables the user to
seamlessly transition to an interactive analysis environment\footnote{i.e. a
Jupyter notebook} primed with the data under investagion. Further details
about the design and capabilities of such a system are presented in
\S\ref{sec:comp:drill}

We suggest that this system should be linked with a metric tracking system:
selected \glspl{aggregate metric} from successive comparable\footnote{i.e.
using the same input data, running with the same configuration} processing
should be tracked as a time series, with the user able to identify outliers
and rapidly switch to the drill-down system to investigate. This capability is
an extension of that already provided by \gls{squash}, and is discussed
further in \S\ref{sec:comp:metric:dashboard}.

The system implementing these capabilities must be able to handle large
datasets --- for example, an entire HSC public data release --- quickly.
Furthermore, we emphasize the importance of ease-of-use for the QA analyst in
order to shorten debugging cycles.

Further, we note that an effective \gls{provenance} system is fundamental to
both drill-down and metric-tracking use cases; we discuss this further in
\S\ref{sec:comp:provenance}.
