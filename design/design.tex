\section{Design sketch}
\label{sec:design}

\subsection{Pipeline debugging}
\label{sec:design:debug}

\assign{John}

The group considering the requirements of Science Pipelines developers for
debugging explored a number of avenues to make developers lives easier. In
doing, so they identified a number of areas in which current systems could be
improved to boost both productivity and developer morale.

In particular, they considered three separate scenarios in which developers
will require support.

\subsubsection{A pipeline is segfaulting}

We considered the following:

\begin{itemize}

\item{A pipeline is failing with a segfault;}
\item{A developer recompiles the failing code with no optimization;}
\item{The unoptimized code is run through a memory analysis tool\footnote{e.g. \href{http://valgrind.org}{Valgind}.};}
\item{This gives some possible locations where arrays are being overrun;}
\item{The unoptimized code is run under \href{https://www.gnu.org/software/gdb/}{GDB};}
\item{This includes the need to start in \href{https://docs.python.org/3/library/pdb.html}{PDB} and attach GDB when the process enters the compiled C++ code;}
\item{Using the debugging utilities, the dev finds where the array is being overrun.}

\end{itemize}


\subsubsection{A pipeline is throwing an exception}

\subsubsection{A \gls{ci} run shows a regression in a \gls{metric value}}



\subsection{Drill down}
\label{sec:design:drill}

\assign{Tim}

\subsection{Drill down}
\label{sec:design:drill}

Drill-down workflows center on the need to quickly and efficiently identify
data processing problems. Typically, these will be identified from
discrepancies identified at higher levels of summary and aggregation.

We therefore envisage a drill-down system which provides rapid retrieval of
relevant quantities (metric values, image cutouts, catalog overlays, etc.)
combined with readily-(re)configurable interactive plotting tools. This is
provided in a browser-based tool which enables the user to rapidly access
successive layers of detail on aggregated metrics (effectively
``de-aggregating'' them on demand), and ultimately enables the user to
seamlessly transition to an interactive analysis environment\footnote{i.e. a
Jupyter notebook} primed with the data under investagion. Further details
about the design and capabilities of such a system are presented in
\S\ref{sec:comp:drill}

We suggest that this system should be linked with a metric tracking system:
selected \glspl{aggregate metric} from successive comparable\footnote{i.e.
using the same input data, running with the same configuration} processing
should be tracked as a time series, with the user able to identify outliers
and rapidly switch to the drill-down system to investigate. This capability is
an extension of that already provided by \gls{squash}, and is discussed
further in \S\ref{sec:comp:squash}.

The system implementing these capabilities must be able to handle large
datasets --- for example, an entire HSC public data release --- quickly.
Furthermore, we emphasize the importance of ease-of-use for the QA analyst in
order to shorten debugging cycles.


\subsection{Datasets and test infrastructure}
\label{sec:design:test}

\assign{John}
