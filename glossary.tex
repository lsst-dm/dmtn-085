%
% Acronyms
%

\newacronym{ci}{CI}{Continuous Integration}
\newacronym{dm}{DM}{Data Management}
\newacronym{hsc}{HSC}{Hyper Suprime-Cam}
\newacronym{kpm}{KPM}{Key Performance Metric}
\newacronym{sdqa}{SDQA}{Science Data Quality Assurance}
\newacronym{qa}{QA}{Quality Assurance}
\newacronym{qawg}{QAWG}{QA Strategy Working Group}

%
% Terminology
%

\newglossaryentry{aggregation}
{
  name={aggregation},
  description={A single result---e.g., a \gls{metric value}---computed from a
  collection of input values. For example, we can sum or average a
  \gls{metric} computed over patches to produce an \gls{aggregate metric} at
  tract level}
}

\newglossaryentry{aggregate metric}
{
  name={aggregate metric},
  description={An \gls{aggregation} of multiple \glspl{point metric}. For
  example, the overall photometric repeatability for a particular tract given
  multiple observations of each star}
}

\newglossaryentry{dashboard}
{
  name={dashboard},
  description={A visual display of the most important information needed to
  achieve one or more objectives, consolidated and arranged on a single screen
  so that the information can be monitored at a glance \citep{Few:2013}}
}

\newglossaryentry{drill down}
{
  name={drill down},
  description={Move from a higher level aggregation of data to its inputs. For
  example, given data describing a tract, we might drill down to constituent
  patches and then to objects; given a visit, we might drill down to CCD and
  then source. In the context of this document, it refers to the act of
  identifying an issue in a high-level summary of the data (e.g. an aberrant
  \gls{metric value}) and interactively investigating its inputs to find the
  source of the problem}
}

\newglossaryentry{gpfs}
{
  name={GPFS},
  description={IBM's General Parallel File System; now known as IBM Spectrum
  Scale. In DM use, this is taken to mean bulk data storage provided through a
  POSIX filesystem interface at the LSST Data Facility}
}

\newglossaryentry{metric}
{
  name={metric},
  description={We follow the \citeds{SQR-019} definition of a metric as a
  measurable quantities which may be tracked. A metric has a name,
  description, unit, references, and tags (which are used for grouping). A
  metric is a scalar by definition. We consider multiple types of metric in
  this document; see \gls{aggregate metric}, \gls{model metric}, \gls{point
  metric}}
}

\newglossaryentry{metric value}
{
  name={metric value},
  description={The result of computing a particular \gls{metric} on some given
  data. Note that we \textit{compute}, rather than measure, metric values}
}

\newglossaryentry{model metric}
{
  name={model metric},
  description={A \gls{metric} describing a model related to the data. For
  example, the coefficients of a 2D polynomial fit to the background of a
  single CCD exposure}
}

\newglossaryentry{monitoring}
{
  name={monitoring},
  description={The process of collecting, storing, aggregating and visualizing
  metrics}
}

\newglossaryentry{point metric}
{
  name={point metric},
  description={A \gls{metric} that is associated with a single entry in a
  catalog. Examples include the shape of a source, the standard deviation of
  the flux of an object detected on a coadd, the flux of an source detected on
  a difference image}
}

\newglossaryentry{releaseable product}
{
  name={releaseable product},
  description={A software package or other component of the DM system which
  is expected to be included in the next tagged release of the system. At time
  of writing, this implies inclusion in a standard top-level package
  (e.g. lsst\_distrib), but we note that future changes to the release procedure
  may render that definition obsolete}
}

\newglossaryentry{squash}
{
  name={SQuaSH},
  description={Science Quality Analysis Harness; \citeds{SQR-009};
  \url{https://squash.lsst.codes}}
}

\newglossaryentry{tidy data}
{
  name={tidy data},
  description={Tidy datasets have a specific structure: each variable is a
  column, each observation is a row, and each type of observational unit is a
  table \citep{JSSv059i10}}
}
