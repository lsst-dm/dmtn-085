\documentclass[DM,authoryear,toc,lsstdraft]{lsstdoc}
\usepackage[nonumberlist,nogroupskip,toc,numberedsection=autolabel]{glossaries}
\usepackage{environ}
\usepackage{enumitem}

\makeglossaries
%
% Acronyms
%

\newacronym{kpm}{KPM}{Key Performance Metric}
\newacronym{qawg}{QAWG}{QA Strategy Working Group}

%
% Terminology
%

\newglossaryentry{aggregation}
{
  name={aggregation},
  description={A single result---e.g., a \gls{metric value}---computed from a
  collection of input values. For example, we can sum or average a
  \gls{metric} computed over patches to produce an \gls{aggregate metric} at
  tract level}
}

\newglossaryentry{aggregate metric}
{
  name={aggregate metric},
  description={An \gls{aggregation} of multiple \glspl{point metric}. For
  example, the overall photometric repeatability for a particular tract given
  multiple observations of each star}
}

\newglossaryentry{dashboard}
{
  name={dashboard},
  description={A visual display of the most important information needed to
  achieve one or more objectives, consolidated and arranged on a single screen
  so that the information can be monitored at a glance \citep{Few:2013}}
}

\newglossaryentry{drill down}
{
  name={drill down},
  description={Move from a higher level aggregation of data to its inputs. For
  example, given data describing a tract, we might drill down to constituent
  patches and then to objects; given a visit, we might drill down to CCD and
  then source. In the context of this document, it refers to the act of
  identifying an issue in a high-level summary of the data (e.g. an aberrant
  \gls{metric value}) and interactively investigating its inputs to find the
  source of the problem}
}

\newglossaryentry{metric}
{
  name={metric},
  description={We follow the \citeds{SQR-019} definition of a metric as a
  measurable quantities which may be tracked. A metric has a name,
  description, unit, references, and tags (which are used for grouping). A
  metric is a scalar by definition. We consider multiple types of metric in
  this document; see \gls{aggregate metric}, \gls{model metric}, \gls{point
  metric}}
}

\newglossaryentry{metric value}
{
  name={metric value},
  description={The result of computing a particular \gls{metric} on some given
  data. Note that we \textit{compute}, rather than measure, metric values}
}

\newglossaryentry{model metric}
{
  name={model metric},
  description={A \gls{metric} describing a model related to the data. For
  example, the coefficients of a 2D polynomial fit to the background of a
  single CCD exposure}
}

\newglossaryentry{monitoring}
{
  name={monitoring},
  description={The process of collecting, storing, aggregating and visualizing
  metrics}
}

\newglossaryentry{point metric}
{
  name={point metric},
  description={A \gls{metric} that is associated with a single entry in a
  catalog. Examples include the shape of a source, the standard deviation of
  the flux of an object detected on a coadd, the flux of an source detected on
  a difference image}
}

\newglossaryentry{releaseable product}
{
  name={releaseable product},
  description={A software package or other component of the DM system which
  is expected to be included in the next tagged release of the system. At time
  of writing, this implies inclusion in a standard top-level package
  (e.g. lsst\_distrib), but we note that future changes to the release procedure
  may render that definition obsolete}
}

\newglossaryentry{tidy data}
{
  name={tidy data},
  description={Tidy datasets have a specific structure: each variable is a
  column, each observation is a row, and each type of observational unit is a
  table \citep{JSSv059i10}}
}


\input meta.tex

\title{QA Strategy Working Group Report}

\author{%
Bellm, E.C.,
Chiang, H.-F.,
Fausti, A.,
Krughoff, K.S.,
MacArthur, L.A.,
Morton, T.D.,
Swinbank, J.D and
Roby, T.
}

\setDocRef{\lsstDocType-\lsstDocNum}
\date{\vcsDate}

\setDocAbstract{%
Abstract.
}

\setDocChangeRecord{%
  \addtohist{\vcsRevision}{\vcsDate}{Unreleased draft.}{Bellm et al.}
}

% Hyperref plays havoc with my crazy recommendation linking TOCs, so disable it.
\hypersetup{linktoc=none}

\makeatletter
\newcommand{\printrecs}{%
  \section{Recommendations}%
  \label{sec:recs}
  \begin{enumerate}[leftmargin=7em,label=QAWG-REC-\arabic*:]%
  \def\@noitemerr{\@latex@warning{Empty objective list}}%
  \@starttoc{rec}%
  \end{enumerate}%
}
\newcommand{\l@rec}[2]{#1}
\makeatother

% Write recommendations so they looks something like this:
%
% \begin{recommendation}{rec:label}{Brief summary}
% Explanatory text, if any.
% \end{recommendation}
\NewEnviron{recommendation}[2]
 {%
  \label{#1}%
  \addcontentsline{rec}{rec}{%
    \noexpand\unexpanded{\unexpanded\expandafter{\item{#2 (\S\ref{#1})}}}%
  }%
  \begin{admonition}{green!5!white}{green!75!black}{Recommendation}{}\emph{#2.} \BODY \end{admonition}%
 }%


\begin{document}

% For tracking who's doing what
\newcommand{\assign}[1]{
  \begin{admonition}{red!5!white}{red!75!black}{Assignee}{}#1\end{admonition}
}

\maketitle


\section{Introduction}
\label{sec:intro}

\assign{John}

This report constitutes the primary artefact produced by the DM \gls{qawg},
addressing its charge as defined in \citeds{LDM-622}.

The complete scope of ``\gls{qa} within \gls{dm}'' is too large to be
coherently addressed by any group on a limited timescale. Per its charge,
then, the \gls{qawg} has constrained itself to considering only those aspects
of \gls{qa} which are most directly relevant to the construction of the LSST
Science Pipelines. In particular, we have considered the tools which
developers need to construct and debug individual algorithms; tools which can
be used to investigate the execution of at-scale pipeline runs; and tools
which can be used to demonstrate that the overall system meets its
requirements (to ``verify'' the system). This deliberately excludes broader
requirements of Commissioning, Science Validation, or run-time
\gls{sdqa}\footnote{Effectively, code executed during prompt or data release
production processing to demonstrate that the data being both ingested and
released is of adequate quality.}.

This report consists broadly of three parts. In \S\ref{sec:approach}, we
describe the approach that the \gls{qawg} took to addressing its charge. In
\S\ref{sec:design}, we present a high-level overview of the systems that we
envisage the future DM comprising. Finally, In \S\ref{sec:comp} we identify
specific components --- which may be pieces of software, documentation,
procedures, or other artifacts --- that should be developed to enable the
capabilities we regard as necessary. In some cases, these components may be
entirely new developments; in others, existing tools developed by the \gls{dm}
subsystem may already be fit for purpose, or can be adapted with some effort.
we have noted when this is the case.

finall,y in Appendix \ref{glo:main} we define a number of key terms which are
used throughout this report and which we we suggest be adopted throughout
\gls{dm} to provide an unambigous vocabulary for referring to QA topics.

\section{Approach to the Problem}
\label{sec:approach}

\assign{John}

\subsection{Pipeline debugging}

\assign{John}

What tools do we need to help pipeline developers with their everyday work?

\begin{itemize}

  \item{How do you go about debugging a \texttt{Task} that is crashing?}
  \item{Is \texttt{lsstDebug} adequate?}
  \item{Do we need an afwFigure, for generating plots, to go alongside \texttt{afwDisplay}, for showing images?}
  \item{What additional capabilities are needed for developers running and debugging at scale, e.g. log collection, identification of failed jobs, etc.}
  \item{What's needed from an image viewer for pipeline developers? Is DS9 or Firefly adequate? Is there value to the afwDisplay abstraction layer, or does it simply make it harder for us to use Firefly's advanced features?}
  \item{How do we view images which don't fit in memory on a single node?}
  \item{How do we handle fake sources? Is this a provenance issue?}

\end{itemize}

\subsection{Drill down}

\assign{John}

\subsection{Datasets and test infrastructure}

\assign{John}

\section{Design sketch}
\label{sec:design}

\subsection{Pipeline debugging}
\label{sec:design:debug}

\assign{John}

\subsection{Drill down}
\label{sec:design:drill}

\assign{Tim}

\subsection{Datasets and test infrastructure}
\label{sec:design:test}

\assign{John}

\section{Core components}

\subsection{Updated pipeline debugging system}
\label{sec:comp:debug}

This component is derived from \S\ref{sec:design:debug}.

The existing pipeline debugging system, lsstDebug, provides useful
capabilities, but --- perhaps due to an idiosyncratic user interface and lack
of documentation --- is frequently not properly exploited. Since the total
amount of code involved in the old system, both implementing it and using it,
is modest, we suggest its wholesale replacement by an alternative, more
approachable, system.

\begin{recommendation}
    {rec:comp:debug}
    {Develop a new pipeline instrumentation and debugging system, replacing lsstDebug}
The total effort expended here should be modest: we expect that design,
implementation and documentation of a new system should take no more than one
full time equivalent month. Converting existing code make take somewhat
longer.
\end{recommendation}

The capabilities of the new system would remain broadly the same as lsstDebug.
Specifically, enabling debugging mode would set a flag within the codebase.
Developers can check for that flag, and take appropriate actions (e.g.
spawning additional plots) if it is set. However, establishing a baseline
expectation of what actions are appropriate or expecting within this debugging
context will require input from the leadership team.

\begin{recommendation}
    {rec:comp:debug:docs}
    {Guidelines for the effective use of the pipeline debugging system should be supplied to developers}
These guidelines should be supplied by the Science Pipelines Product Owners
and the LSST Software Architect, and made available through the DM Developer
Guide \citeds{DevGuide}.
\end{recommendation}

Developers and users rightly expect that the debugging system will provide an
idiomatic user interface. We suggest that the complexity of the existing
lsstDebug interface has been a bar to adoption. We therefore propose that
simplicity be key aim of the new system. In that vein, we suggest that the
debugging system should be controlled by a single boolean parameter, with no
further user-driven customization, and with very explicit documentation. We
acknowledge that this is a direct tradeoff between granularity and simplicity,
and suggest that history drives us to err on the side of simplicity.

\begin{recommendation}
    {rec:comp:debug:use}
    {Debugging mode shoud be binary: it is either enabled or disabled, with no further configuration}
    For example, within a \texttt{Task}, one might check if debugging mode is
    enabled with a statement like \texttt{if self.config.debug: ...}.
\end{recommendation}

\begin{note}
As of 2018-08-21, I (JDS) have removed references to enabling debugging
through the task config system. This is based on OOB discussions that the
config is most appropriate for recording options which affect the science
outputs, whereas (one assumes) that the code should generate the same science
results whether or not debugging is enabled. I think that means it really
wants to be driven by a command line argument and a set of explicit arguments,
but I am happy to be persuaded otherwise.
\end{note}

The natural consequence of this is that debugging of tasks involved by some
form of middleware (e.g. \texttt{CmdLineTask}, \texttt{PipelineTask} called by
an executor) should receive an explicit argument enabling debugging. All other
callables which support debugging must take an explict argument which enables
it.

\subsection{Logging}
\label{sec:comp:log}

This component is derived from \S\ref{sec:design:debug}.

Logging is an important aspect of running large data processing.  It is also
integral to quality assessment as the logging information provides important
contextual information when inspecting data for quality issues.  Specifically,
log messages presenting information about the processing: e.g. PSF width,
number of stars used in a model fit, can indicate problems with the
algorithmic behavior or input data.  Logging also provides information about
potential causes for unexpected termination including exit codes and
exceptions.

DM already provides a logging system (the ``log'' package), and the API
documentation provided for
it\footnote{\url{https://developer.lsst.io/stack/logging.html}} is
adequate\footnote{It is, perhaps, worth noting that configuring the output of the
logging system is much less straightforward and is badly documented:
\url{http://doxygen.lsst.codes/stack/doxygen/x_masterDoxyDoc/log.html\#configuration}.}.

However, the outputs of the logging system become increasingly hard to parse
as the number of concurrent processes increases: a common complaint when
running large-scale data processing is that it's difficult to identify
failures and understand where they came from.

\begin{recommendation}
  {rec:comp:log}
  {A log aggregation and monitoring service should be provided for large-scale processing jobs at the Data Facility}
Such a service should not be a requirement for jobs to execute (in particular,
when running in non-Data Facility environments, logs should continue to be
generated and collected as at present).
\end{recommendation}

Log aggregation should provide the following capabilities:

\begin{itemize}
\item{Display all log messages from a given pipeline execution;}
\item{Display messages at a selected log level (\texttt{INFO}, \texttt{WARN}, \texttt{ERROR}, etc);}
\item{Display time-ordered logs for a given thread;}
\item{Display logs, exit status and/or exceptions for threads ending in an unexpected state;}
\item{Searchable per process for regexp and timestamp.}
\end{itemize}

We believe that these capabilities could be provided by building atop
off-the-shelf log aggregation tools such as
Logstash\footnote{\url{https://www.elastic.co/products/logstash}}. However, we
suggest that exploring synergies with existing or planned Data Facility
services is likely essential. We emphasize that effective log management
capabilities are of increasing importance not just in the operational era but
in the immediate future, as we move towards large-scale data processing in
support of commissioning: deploying a basic version of this service should be
regarded as a high priority.

\subsection{Workload Management System}
\label{sec:comp:workload}

\assign{Simon}

I believe we aggreed that if the logging system and provenance system are sufficient to meet recommendations, this aspect is essentially a solved problem.

I do think that it also depends on the orchestration layer, but I don't think that can be in the scope of this group.

Derived from \S\ref{sec:design:debug}.

\subsection{Visualization}
\label{sec:comp:vis}

Derived from \S\ref{sec:design:debug}.

Broadly, we regard ``visualization'' as an umbrella term covering both visualizations derived from catalogs as well as image display and manipulation.
The concerns expressed by developers and others are, on the whole, common to both; the solutions may not be.

For both visualization regimes, the predominant request from developers is that the project provide them with clear guidance as to both what resources will be provided and supported by the LSST construction effort (for example, tools like Firefly or abstractions like \texttt{afwDisplay}) and which tools they are required or expected to use in the interest of maintaining a coherent and consistent codebase and set of outputs.

In this section, we concentrate on ad-hoc visualization in support of the regular pipeline developer.
In \S\S\ref{sec:design:drill} and \ref{sec:comp:drill}, we describe the design of an interactive ``drill-down'' tool, which will provide a number of plotting and data exploration capabilities.
It is our expectation that, however comprehensive such a tool might become, there will always be a necessity for individual developers to be able to quickly investigate their data in as flexible a way as possible; conversely, wherever practical we should enable developers to exploit, and encourage them to remain consistent with, the capabilities and conventions delivered by the drill-down tool.
These sections should therefore be regarded as complementary.

\subsubsection{Catalog visualization}
\label{sec:comp:vis:catalog}

The group notes that there are many contexts in which visualizations derived from catalogs might be required (for example, in-line display in a Jupyter notebook, persisting plots from a debugging session, as described in \S\ref{sec:comp:debug}, preparing plots for publication, etc), which may all have substantially different requirements.
We also note that there exists a diverse infrastructure of scientific plotting and data exploration tools in the Python community, a comprehensive selection of which has been collated by the PyViz project\footnote{\url{http://pyviz.org}}, which also provides documentation on effectively using them in conjunction with each other.
Given that, we regard it as unnecessary, and indeed undesirable, for LSST to attempt to standardize on any particular plotting tool: we should rather encourage developers to exploit community resources with minimal overhead.

\begin{recommendation}
  {rec:comp:vis:catalog:pyviz}
  {DM should formally adopt the PyViz ecosystem}
This adoption would include, for example, including PyViz tools in a regular installation of the LSST Stack; providing training and documentation for developers and --- crucially --- developing interfaces which enable LSST conventions (afw tables, the Data Butler) to be used in the PyViz context.
\end{recommendation}

Many visualization use cases involve manipulating data at a larger scale than can conveniently be done on a single compute node.
Within the PyViz ecosystem, Dask\footnote{http://dask.pydata.org/en/latest/} is the preferred approach.
We note that there many be some redundancy between Dask and the LSST middleware (Butler, PipelineTask and appropriate executors, etc).
However, Dask provides a convenient, easy to install and use solution which can immediately address developer needs.

\begin{recommendation}
  {rec:comp:vis:catalog:dask}
  {DM should adopt Dask to enable users to work with larger than memory data}
This might be achieved by providing users with the ability to spin up Dask clusters on demand using (say) Kubernetes, or by providing a Dask cluster at the LSST Data Facility to which users can connect.
If ongoing middleware development renders this obsolete, then it can be retired.
\end{recommendation}

\subsubsection{Image visualization}
\label{sec:comp:vis:image}

\subsection{Image viewer}
\label{sec:comp:image}

\assign{Trey}

Derived from \S\ref{sec:design:debug}.

As of 2018-06-12 we haven't converged on a solid recommendation here.

Key considerations:

\begin{itemize}

  \item{Firefly is the annointed solution being provided by DM to external
  stakeholders (commissioning, operations, etc). It feels right to everybody
  that we should be dogfooding it, and also benefitting from development being
  carried out for those stakeholders.}

  \item{Currently, Firefly is unappealing to developers (primarily, I think,
  because of slowness of user interface, and perhaps also due to installation
  issues). Can we resolve these issues?}

  \item{We'd want to support visualization in a number of different
  environments, for e.g.:

    \begin{itemize}

      \item{Inside a Jupyter notebook;}
      \item{As a standalone tool, \`a la DS9;}
      \item{Embedded in a \gls{dashboard}, \`a la JS9, Aladin-Lite, etc.}

    \end{itemize}
  }

  \item{Do we lose flexibility by mandating the use of a backend-agnostic API
  (\texttt{afwDisplay}) rather than going ``all-in'' on e.g. a custom Firefly
  interface?}

  \item{We'll need to do full focal plane visualization, which none(?) of the
  current tools support well.}

\end{itemize}

Options include:

\begin{itemize}

  \item{Do nothing; continue as we are, which means most people will use DS9
  and a few will drift to Firefly as commissioning ramps up.}

  \item{Issue some sort of edict that pipelines developers have to use
  Firefly.}

  \item{Encourage the use of some other tool (Ginga?) instead of or as well as
  some of the above.}

  \item{Probably others.}

\end{itemize}

Sounds like we need somebody from the QAWG to actually write some requirements
--- or a wishlist set of features we want --- here.


\subsection{Catalog visualization tools}

\assign{Lauren}

Derived from \S\ref{sec:design:debug}.

For visualizing bigger-than-memory catalogs. May include e.g. the capability
to spin up Dask clusters on demand, combined with
Holoviews/Datashader/whatever. Somebody who knows about this stuff needs to
write a summary...

\subsection{Provenance}

\assign{Hsin-Fang}

The QAWG recognizes the importance of provenance and the
implementation of the provenance system will impact QA work
significantly.  We recommend high priority to finalize the design
and implementation of the provenance system.

The QAWG believes that the requirements on provenance tracking are
adequate as described in the Data Management Middleware Requirements
(\citeds{LDM-556}) and Data Management Data Backbone Services
Requirements (\citeds{LDM-635}). QA use cases provide no further
requirements.

With the current design of the Gen 3 Middleware, each dataset
can be linked to provenance information such as input datasets,
pipeline definition, configurations, and software version.
QC metric values will be Butler datasets so they will be associated
with specific data IDs, and their provenance can be traced like
other datasets.  The Gen3 Butler/SuperTask generated provenance is
the primary source of provenance in the QA use cases as the
Butler/SuperTask framework is responsible of producing provenance
in the Gen 3 Middleware design.  For production runs, the Data
Backbone Services may store additional provenance, for example from
the Batch Production Services, besides the Butler/SuperTask generated
provenance.

Regarding provenance of the database records, the QAWG does not
place a strong requirement on per-source provenance.  The current
design is that each database record in the production database is
either ingested from a file, of which the full provenance is traced,
or from an uniquely identifiable execution.  This design does not
directly provide detailed per-source provenance, such as what exact
input images acually contribute to the measurement of a particular
source, but the full inputs that can contribute to the source.  The
provenance tracking of synthetic sources is also unclear.  The QAWG
thinks most low-level information can be uncovered through the drill
down system (\S\ref{sec:design:drill}).  Diagnostic information can
also be computed in the pipeline codes and stored as additional
columns.  We recommend the tooling to investigate the composition
of coadds be made in the drill down system and does not put this
requirement in the provenance system.

The QAWG notes that some high level aggregate data products that
are derived from provenance data can be useful in QA work.  For
example, the number of images that contribute to each coadd patch
can be obtained from provenance data.  However, the QAWG are not
immediately convinced that it's worth spending significant time
investigating what aggregate products need to be considered.

Derived from \S\ref{sec:design:debug}.

This section should note:

\begin{itemize}

  \item{That provenance is an immediate issue impacting QA work, so a solution
  is a priority;}

  \item{Some requirements as to the granularity at which provenance tracking
  is necessary for QA.}

\end{itemize}

\subsection{Documentation content updates}
\label{sec:comp:doc}

Derived from \S\ref{sec:design:test}.

\assign{John}

\begin{itemize}

  \item{Clearer guidance on unit tests.}
  \item{Clearer guidance on code review, with requirements for test coverage
  etc.}

\end{itemize}

\subsection{Testing for documentation}

Derived from \S\ref{sec:design:test}.

\assign{John}

\begin{itemize}

  \item{Examples.}

\end{itemize}

\subsection{CI system updates}

Derived from \S\ref{sec:design:test}.

\assign{John}

\begin{itemize}

  \item{Test coverage.}
  \item{Tighter control of the environment.}
  \item{Better notifications.}
  \item{Better descriptions of which jobs do what.}
  \item{Clear description of what Developers are required to do before merging
  to master (see also \S\ref{sec:comp:doc}).}

\end{itemize}


\subsection{Metrics Dashboard / SQuaSH}

Derived from \S\ref{sec:design:test}.

\assign{Angelo}

Broadly as current SQuaSH, but to track:

\begin{itemize}

  \item{Code execution time.}
  \item{Test coverage?}
  \item{Notifications of regressions.}

\end{itemize}

\subsection{Standard format dataset package}
\label{sec:comp:lfs-dataset}

Derived from \S\ref{sec:design:test}.

\assign{Hsin-Fang}

The standard format of a dataset package is a ready-to-use Butler
repository and follows the format of a Butler repository as defined
in its corresponding obs package.  The format is configurable by
design, however, it is tied to the codes in the stack, so can change
from a software stack version to another.  Besides implementations
in the obs packages and Butler, other evolvement in the software
stack, such as handling of calibration data and reference catalog,
can also make a once-working repository incompatible.  Therefore,
maintenance is occassionally needed to ensure the usability of a
dataset package.  The QAWG recommends a per-dataset product owner.

The Obs Pkg WG \jira{RFC-393} is charged to re-design and refactor
the obs packages for maintainability and extensibility. We suggest
the Obs Pkg WG take into considerations in their design to mitigate
the close tie between a Butler repository and its obs package
implmentations, as well as adopt a common structure across different
cameras when possible.  After the refactoring, the obs packages
will rarely change so the dataset format will be more stable.  The
QAWG recommends prioritise the Obs Pkg WG.

In some cases, a dataset pacakge may contain additional data not
in the format of a Butler repository. We recommend the format as
described in DM Developer Guide Common Dataset Organization and
Policy \footnote{\url{https://developer.lsst.io/services/datasets.html}}
and expand the policy as needed.

\subsection{Standard test package design}

Derived from \S\ref{sec:design:test}.

\assign{Hsin-Fang}

Should address the union of lsst\_dm\_stack\_demo, ci\_hsc, validate\_drp use
cases.

\subsection{Updates to guidelines for GPFS-based dataset storage}
\label{sec:comp:gpfs-dataset}


\appendix
\printrecs
\glsaddall
\renewcommand*{\glsautoprefix}{glo:}
\printglossary[style=index,numberedsection=autolabel]


\bibliography{lsst,lsst-dm,refs_ads,refs,books}

\end{document}
