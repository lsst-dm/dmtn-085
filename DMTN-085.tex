\documentclass[DM,authoryear,toc,lsstdraft]{lsstdoc}
\usepackage[nonumberlist,nogroupskip,toc,numberedsection=autolabel]{glossaries}
\usepackage{environ}
\usepackage{enumitem}

% Inconsolata is used by lsst-texmf, but doesn't have an italic variant. Fake one.
\setmonofont[AutoFakeSlant]{Inconsolata}

\makeglossaries
%
% Acronyms
%

\newacronym{kpm}{KPM}{Key Performance Metric}
\newacronym{qawg}{QAWG}{QA Strategy Working Group}

%
% Terminology
%

\newglossaryentry{aggregation}
{
  name={aggregation},
  description={A single result---e.g., a \gls{metric value}---computed from a
  collection of input values. For example, we can sum or average a
  \gls{metric} computed over patches to produce an \gls{aggregate metric} at
  tract level}
}

\newglossaryentry{aggregate metric}
{
  name={aggregate metric},
  description={An \gls{aggregation} of multiple \glspl{point metric}. For
  example, the overall photometric repeatability for a particular tract given
  multiple observations of each star}
}

\newglossaryentry{dashboard}
{
  name={dashboard},
  description={A visual display of the most important information needed to
  achieve one or more objectives, consolidated and arranged on a single screen
  so that the information can be monitored at a glance \citep{Few:2013}}
}

\newglossaryentry{drill down}
{
  name={drill down},
  description={Move from a higher level aggregation of data to its inputs. For
  example, given data describing a tract, we might drill down to constituent
  patches and then to objects; given a visit, we might drill down to CCD and
  then source. In the context of this document, it refers to the act of
  identifying an issue in a high-level summary of the data (e.g. an aberrant
  \gls{metric value}) and interactively investigating its inputs to find the
  source of the problem}
}

\newglossaryentry{metric}
{
  name={metric},
  description={We follow the \citeds{SQR-019} definition of a metric as a
  measurable quantities which may be tracked. A metric has a name,
  description, unit, references, and tags (which are used for grouping). A
  metric is a scalar by definition. We consider multiple types of metric in
  this document; see \gls{aggregate metric}, \gls{model metric}, \gls{point
  metric}}
}

\newglossaryentry{metric value}
{
  name={metric value},
  description={The result of computing a particular \gls{metric} on some given
  data. Note that we \textit{compute}, rather than measure, metric values}
}

\newglossaryentry{model metric}
{
  name={model metric},
  description={A \gls{metric} describing a model related to the data. For
  example, the coefficients of a 2D polynomial fit to the background of a
  single CCD exposure}
}

\newglossaryentry{monitoring}
{
  name={monitoring},
  description={The process of collecting, storing, aggregating and visualizing
  metrics}
}

\newglossaryentry{point metric}
{
  name={point metric},
  description={A \gls{metric} that is associated with a single entry in a
  catalog. Examples include the shape of a source, the standard deviation of
  the flux of an object detected on a coadd, the flux of an source detected on
  a difference image}
}

\newglossaryentry{releaseable product}
{
  name={releaseable product},
  description={A software package or other component of the DM system which
  is expected to be included in the next tagged release of the system. At time
  of writing, this implies inclusion in a standard top-level package
  (e.g. lsst\_distrib), but we note that future changes to the release procedure
  may render that definition obsolete}
}

\newglossaryentry{tidy data}
{
  name={tidy data},
  description={Tidy datasets have a specific structure: each variable is a
  column, each observation is a row, and each type of observational unit is a
  table \citep{JSSv059i10}}
}


\input meta.tex

\title{QA Strategy Working Group Report}

\author{%
Bellm, E.C.,
Chiang, H.-F.,
Fausti, A.,
Krughoff, K.S.,
MacArthur, L.A.,
Morton, T.D.,
Swinbank, J.D and
Roby, T.
}

\setDocRef{\lsstDocType-\lsstDocNum}
\date{\vcsDate}
\setDocUpstreamLocation{\url{https://github.com/lsst-dm/dmtn-085 }}

\setDocAbstract{%
This document describes the work undertaken by the QA Strategy Working Group and presents its conclusions as a series of recommendations to DM Project Management.
}

\setDocChangeRecord{%
  \addtohist{\vcsRevision}{\vcsDate}{Unreleased draft.}{Swinbank}
}

% Hyperref plays havoc with my crazy recommendation linking TOCs if we let it
% turn the section name into a hyperlink, so we use the page numbers instead.
\hypersetup{linktoc=page}

\makeatletter
\newcommand{\printrecs}{%
  \section{Recommendations}%
  \label{sec:recs}
  \begin{enumerate}[leftmargin=7em,label=QAWG-REC-\arabic*:]%
  \def\@noitemerr{\@latex@warning{Empty objective list}}%
  \@starttoc{rec}%
  \end{enumerate}%
}
\newcommand{\l@rec}[2]{#1}
\newenvironment{recbox}
{
  \begin{center}
  \begin{minipage}[h]{.85\linewidth}
    \begin{tcolorbox}[colback=green!5!white,colframe=green!75!black,title=\textbf{Recommendation}]
}
{
    \end{tcolorbox}
  \end{minipage}
  \end{center}
}

\makeatother

% Write recommendations so they looks something like this:
%
% \begin{recommendation}{rec:label}{Brief summary}
% Explanatory text, if any.
% \end{recommendation}
\NewEnviron{recommendation}[2]
 {%
  \label{#1}%
  \addcontentsline{rec}{rec}{%
    \noexpand\unexpanded{\unexpanded\expandafter{\item{#2 (\S\ref{#1})}}}%
  }%
  \begin{recbox}
  \emph{#2.} \par \BODY%
  \end{recbox}
 }%


\begin{document}

\maketitle

\section{Introduction}
\label{sec:intro}

This report constitutes the primary artifact produced by the DM \gls{qawg},
addressing its charge as defined in \citeds{LDM-622}.

It is worth starting by revisiting the definition of Quality
\textit{Assurance}, or \gls{qa}.  In particular, note that \citeds{LDM-522}
defines QA as ``Quality \textit{Analysis}'', a process which is undertaken by
humans during commissioning and operations, and which stands in contrast to
automated Quality Control (\gls{qc}) systems. For the purposes of this group,
we have taken a more holistic definition (following the guidance in the
charge) of \gls{qa}, covering all activities undertaken by the \gls{dm}
construction project to ensure the ultimate quality of its deliverables.

The complete scope of ``\gls{qa} within \gls{dm}'' is too large to be
coherently addressed by any group on a limited timescale. Per its charge,
then, the \gls{qawg} has constrained itself to considering only those aspects
of \gls{qa} which are most directly relevant to the construction of the LSST
Science Pipelines. In particular, we have considered the tools which
developers need to construct and debug individual algorithms; tools which can
be used to investigate the execution of pipeline runs at scales beyond those
which are trivially addressable by individual developers on single compute
systems; and tools which can be used to demonstrate that the overall system
meets its requirements (to ``verify'' the system). This deliberately excludes
broader requirements of Commissioning, Science Validation, or run-time
\gls{sdqa}\footnote{Effectively, code executed during prompt or data release
production processing to demonstrate that the data being both ingested and
released is of adequate quality.}.

This report consists broadly of three parts. In \S\ref{sec:approach}, we
describe the approach that the \gls{qawg} took to addressing its charge. In
\S\ref{sec:design}, we present a high-level overview of the systems that we
envisage the future DM comprising. Finally, in \S\ref{sec:comp} we identify
specific components --- which may be pieces of software, documentation,
procedures, or other artifacts --- that should be developed to enable the
capabilities we regard as necessary. In some cases, these components may be
entirely new developments; in others, existing tools developed by the \gls{dm}
subsystem may already be fit for purpose, or can be adapted with some effort.
We have noted when this is the case.

Throughout, we provide a number of recommendations which we suggest should be
adopted by the \gls{dm} Subsystem as a whole. These recommendations identify
specific actions that should be taken or capabilities that should be provided;
in general, addressing them will require action by the Project Manager or
T/CAMs to schedule appropriate activities.

Finally, in Appendix \ref{glo:main}, we define a number of key terms which are
used throughout this report and which we suggest be adopted across \gls{dm} to
provide an unambiguous vocabulary for referring to QA topics.

\begin{recommendation}
    {rec:glossary}
    {Adopt the definitions of QA-related terms in the \citeds{DMTN-085} glossary subsystem-wide}
For example, by inclusion in a subsystem-level glossary; refer to
\jira{DM-9807}, \jira{DM-14877}, and \jira{DM-14911}.
\end{recommendation}

\section{Approach to the Problem}
\label{sec:approach}

The QAWG addressed its charge by sub-dividing the problem space into three separate areas:

\begin{itemize}

  \item{Addressing the needs of developers writing and debugging algorithms on the small scale;}
  \item{Developing tooling to address the \textit{\gls{drill down}} use case;}
  \item{Providing the infrastructure needed to support automatic testing and verification.}

\end{itemize}

Each of these areas were assigned to a separate sub-group within the WG for
brainstorming and developing approaches, with each sub-group regularly
reporting progress to overall working group meetings.

When each sub-group had developed a strong concept for the tooling needed to
address their particular part of the charge, the whole working group reviewed
each design in detail, identifying and developing specifications for common
components or activities that enable one or more of the designs.

In \S\S\ref{sec:approach:debug}, \ref{sec:approach:drill} and
\ref{sec:approach:test}, we provide details about the charge provided to each
sub-group.

\subsection{Pipeline debugging}
\label{sec:approach:debug}

What tools do we need to help pipeline developers with their every-day work?
Specifically:

\begin{itemize}

  \item{How do you go about debugging a \texttt{Task} that is crashing?}
  \item{Is \texttt{lsstDebug} adequate?}
  \item{Do we need an \texttt{afwFigure}, for generating plots, to go alongside \texttt{afwDisplay}, for showing images?}
  \item{What additional capabilities are needed for developers running and debugging at scale, e.g. log collection, identification of failed jobs, etc.}
  \item{What's needed from an image viewer for pipeline developers? Is DS9 or Firefly adequate? Is there value to the afwDisplay abstraction layer, or does it simply make it harder for us to use Firefly's advanced features?}
  \item{How do we view images which don't fit in memory on a single node?}
  \item{How do we handle fake sources? Is this a provenance issue?}

\end{itemize}

\subsection{Drill down}
\label{sec:approach:drill}

How can we provide developers and testers with the ability to ``drill down''
from high level aggregated metrics to explore the source data and intermediate
data products that contributed to them? Specifically:

\begin{itemize}

  \item{What sort of metrics should be extracted from running pipelines\footnote{Scalars, vectors, spatially binned quantities, etc.}?}
  \item{How can those metrics be displayed on a dashboard? Is a simple time-series adequate, or do we need other types of plotting?}
  \item{By what mechanism can the user drill-down from those aggregated metrics to identify the sources of problems? Do they click through pre-generated plots, or jump straight into a notebook environment?}
  \item{Assuming the user ends up in an interactive environment, what are its capabilities?}
  \item{What do the above tell us about the data products that pipelines need to persist (both in terms of metrics that are posted to \gls{squash}, and regular pipeline outputs, Parquet tables, HDF5 files, etc)?}

\end{itemize}

\subsection{Datasets and test infrastructure}
\label{sec:approach:test}

What infrastructure must we make available to enable testing and verification
of the DM system? Specifically:

\begin{itemize}

  \item{Are any changes needed to the way that DM currently handles unit testing?}
  \item{How are datasets made available to developers? Git LFS repositories?  \gls{gpfs}?}
  \item{What is the appropriate cadence for running small/medium/large scale integration tests and reprocessing of known data?}
  \item{How is the system for tracking \glspl{metric} managed? --- how are the metric calculation jobs run? By whom? How often?}
  \item{How run-time performance of the science algorithms be tracked?}

\end{itemize}

\section{Design sketch}
\label{sec:design}

In this section, we summarize the issues identified and approaches suggested
by the groups described in \S\ref{sec:approach}. From these, we synthesize a
number of concrete actions --- tools to be developed, documentation to be
provided, etc --- which we recommend to the project in \S\ref{sec:comp}.

\subsection{Pipeline debugging}
\label{sec:design:debug}

The group does not identify a single, over-arching tool or concept which would
solve the problem of developer productivity and happiness. Instead, we believe
that developer needs can be best addressed by making incremental improvements
to a number of core pieces of \gls{dm} technology and infrastructure which
team members regularly interact with. In particular, we identified the
following major ``pain points'' for developers:

\begin{itemize}

  \item{The pipeline debugging system, \texttt{lsstDebug}, is badly documented
  and awkward to use, and developers lack appropriate guidelines on how embed
  debugging information into algorithms or on how to use that information to
  most effectively debug running pipelines.}

  \item{Developers find it hard to know how to debug their code when it is
  running at scale. Issues include parsing logs when a large number of
  concurrent processes are running; inadequate documentation of the existing
  Slurm system; uncertainty about replacement of Slurm with a future workflow
  system; and difficulties in understanding the \gls{provenance} of data
  products.}

  \item{The project has issued unclear guidance and inadequate documentation
  to developers about the appropriate tools to be used for visualizing data.
  This is most pronounced for image visualization\footnote{Developers have the
  impression they ``ought to'' be using Firefly, but there is much uncertainty
  around its suitability for the task and its future development direction.},
  but also applies to catalog data.}

  \item{Developers struggle to identify suitable datasets for running tests
  --- both small and large scale --- in a convenient form. Large repositories
  exist on the project \gls{gpfs} system, but it's unclear what they contain
  or how to effectively access them\footnote{A canonical example is the
  \gls{hsc} public data release: the volume of data is overwhelming for a
  developer who simply needs some representative \gls{hsc} data to test an
  algorithm.}; smaller repositories exist on GitHub\footnote{e.g.
  \href{https://github.com/lsst/validation_data_hsc}{validation\_data\_hsc},
  \href{https://github.com/lsst/afwdata}{afwdata},
  \href{https://github.com/lsst/ap_verify_hits2015}{ap\_verify\_hits2015},
  etc.}, but they are inconsistent in structure, content and documentation:
  it's impossible for a developer to quickly identify data which is relevant
  to their use case, or to establish whether some particular reduction of the
  data is ``correct''. Instead, they rely on folklore and talking to peers to
  find data that ``worked for somebody else'', with (often) predictably
  frustrating results.}

\end{itemize}

To address these issues, the working group suggests the development of a
number of separate-but-related components. These include:

\begin{itemize}

  \item{An updated system for instrumenting running pipeline code;
  effectively, a revision of \texttt{lsstDebug}. This is developed further in
  \S\ref{sec:comp:debug}.}

  \item{A revised set of tooling for generating, aggregating and analyzing
  logs. This is developed further in \S\ref{sec:comp:log}.}

  \item{Revised documentation on interacting with the workload management
  system. This is developed further in \S\ref{sec:comp:workload}.}

  \item{Guidelines for the structure and maintenance of data repositories.
  This is developed further in \S\ref{sec:comp:dataset}.}

  \item{A clear roadmap for the development of visualization tools, and,
  derived from that, guidelines on how to apply those in the development of
  pipelines. This is developed further in \S\ref{sec:comp:vis}.}

\end{itemize}

In addition, the WG suggests prioritization of developer-accessible
systems for tracking and understanding the \gls{provenance} of pipeline
outputs. For additional comments on provenance from a QA perspective, refer to
\S\ref{sec:comp:provenance}.

\subsection{Drill down}
\label{sec:design:drill}

Drill-down workflows center on the need to quickly and efficiently identify
data processing problems. Typically, these will be identified from
discrepancies identified at higher levels of summary and aggregation.

We therefore envisage a drill-down system which provides rapid retrieval of
relevant quantities (metric values, image cutouts, catalog overlays, etc.)
combined with readily-(re)configurable interactive plotting tools. This is
provided in a browser-based tool which enables the user to rapidly access
successive layers of detail on aggregated metrics (effectively
``de-aggregating'' them on demand), and ultimately enables the user to
seamlessly transition to an interactive analysis environment\footnote{i.e. a
Jupyter notebook} primed with the data under investagion. Further details
about the design and capabilities of such a system are presented in
\S\ref{sec:comp:drill}

We suggest that this system should be linked with a metric tracking system:
selected \glspl{aggregate metric} from successive comparable\footnote{i.e.
using the same input data, running with the same configuration} processing
should be tracked as a time series, with the user able to identify outliers
and rapidly switch to the drill-down system to investigate. This capability is
an extension of that already provided by \gls{squash}, and is discussed
further in \S\ref{sec:comp:squash}.

The system implementing these capabilities must be able to handle large
datasets --- for example, an entire HSC public data release --- quickly.
Furthermore, we emphasize the importance of ease-of-use for the QA analyst in
order to shorten debugging cycles.

\subsection{Datasets and test infrastructure}
\label{sec:design:test}

The group regards this part of its charge as qualitatively different from the
other two (\S\S\ref{sec:design:debug} \& \ref{sec:design:drill}): where those
focus on servicing the needs of the individual developer or analyst, this
material describes the capabilities that are required by the overall subsystem
to track its progress and verify its deliverables.

We considered the following three aspects of this part of the charge:

\subsubsection{Small-scale unit and documentation tests}

The current unit test and \gls{ci} system functions well. However, we suggest
a number of improvements. These include clearer instructions to developers on
the expectations for testing (discussed in \S\ref{sec:comp:code_quality}) and
assorted minor upgrades to the \gls{ci} system, primarily to enable more
convenient failure notifications and tighter control of the test environment
(\S\ref{sec:comp:ci}).

The single highest priority we identified was the lack of a current system for
performing CI on example code, documentation and (perhaps) Jupyter notebooks.
This has already rendered many of the examples provided in the current
codebase obsolete --- and because they aren't regularly tested, we have no way
to know what else is broken and will fail or otherwise confuse new (or
existing) users. We regard addressing this as one of the single highest
priorities for the project. It is discussed further in
\S\ref{sec:comp:doctest}.

\subsubsection{Integration tests}
\label{sec:design:test:integration}

DM's integration test needs are currently served by a heterogeneous mix of
approaches: from ``unit'' tests which effectively test the integration of
multiple packages, to explicit integration jobs executed by the CI
system --- of which there are multiple types, and no common
implementation standard\footnote{See
\href{https://github.com/lsst/lsst_dm_stack_demo}{lsst\_dm\_stack\_demo},
\href{https://github.com/lsst/ci_hsc}{ci\_hsc},
\href{https://github.com/lsst/validate_drp}{validate\_drp}, etc.} --- to large
scale data processing exercises undertaken periodically at the
\gls{ldf}\footnote{e.g.
\url{https://confluence.lsstcorp.org/display/DM/Reprocessing+of+the+HSC+RC2+dataset}.}.

We posit that a unified and documented approach to integration testing will
provide lower implementation overheads, fewer surprises for developers, and
more predictable coverage for the project. We expand on the considerations and
design for such a system in \S\ref{sec:comp:test_pkg}.

Developing a standardized approach to integration testing requires a
standardized approach to the management of test datasets: this is addressed in
\S\ref{sec:comp:dataset}. It also impacts upon the \gls{ci} system, and hence
is relevant to \S\ref{sec:comp:ci}.

Finally, we note that the way in which pipelines are defined and executed will
likely evolve rapidly over the remainder of the construction period as
technologies like PipelineTask and Butler Generation 3 come into use. Given
the rapidly-moving nature of this work, the QAWG regards them as out of scope,
except insofar as we urge that QA tasks should be implemented and kept up to
date with these new frameworks as they become available.

\subsubsection{Metric and performance tracking}
\label{sec:design:test:metrics}

DM already has a metric-tracking system based on \gls{squash}. This
effectively tracks the time evolution of a number of \glspl{aggregate metric}
based on data processing under controlled conditions (effectively, the results
of selected integration tests, carried out as per
\S\ref{sec:design:test:integration}). This basic machinery is excellent, but
has suffered from low adoption among DM developers, and it is not clear that
any regular checking of or acting upon trends in the calculated metrics is
being undertaken.

We therefore propose a series of updates and refinements to \gls{squash},
focusing on a series of modest enhancements to its capabilities to alert DM
developers and management to regressions (or improvements) in selected
performance metrics. These are developed in \S\ref{sec:comp:metric:dashboard}.

The primary means by which data is submitted to \gls{squash} is through the
\href{https://github.com/lsst/verify}{\texttt{lsst.verify}} framework and
associated metric definitions. The mechanisms by which this framework is
integrated with the Science Pipelines are unclear (refer to \citeds{DMTN-057}
for a discussion) and it has never undergone a design review or acceptance
process.  Clearer ownership and direction for this part of the system are
essential to drive adoption. We discuss this in more detail and provide
suggestions in \S\ref{sec:comp:metric:collect}.


\section{Core components}
\label{sec:comp}

\subsection{Updated pipeline debugging system}
\label{sec:comp:debug}

This component is derived from \S\ref{sec:design:debug}.

The existing pipeline debugging system, lsstDebug, provides useful
capabilities, but --- perhaps due to an idiosyncratic user interface and lack
of documentation --- is frequently not properly exploited. Since the total
amount of code involved in the old system, both implementing it and using it,
is modest, we suggest its wholesale replacement by an alternative, more
approachable, system.

\begin{recommendation}
    {rec:comp:debug}
    {Develop a new pipeline instrumentation and debugging system, replacing lsstDebug}
The total effort expended here should be modest: we expect that design,
implementation and documentation of a new system should take no more than one
full time equivalent month. Converting existing code make take somewhat
longer.
\end{recommendation}

The capabilities of the new system would remain broadly the same as lsstDebug.
Specifically, enabling debugging mode would set a flag within the codebase.
Developers can check for that flag, and take appropriate actions (e.g.
spawning additional plots) if it is set. However, establishing a baseline
expectation of what actions are appropriate or expecting within this debugging
context will require input from the leadership team.

\begin{recommendation}
    {rec:comp:debug:docs}
    {Guidelines for the effective use of the pipeline debugging system should be supplied to developers}
These guidelines should be supplied by the Science Pipelines Product Owners
and the LSST Software Architect, and made available through the DM Developer
Guide \citeds{DevGuide}.
\end{recommendation}

Developers and users rightly expect that the debugging system will provide an
idiomatic user interface. We suggest that the complexity of the existing
lsstDebug interface has been a bar to adoption. We therefore propose that
simplicity be key aim of the new system. In that vein, we suggest that the
debugging system should be controlled by a single boolean parameter, with no
further user-driven customization, and with very explicit documentation. We
acknowledge that this is a direct tradeoff between granularity and simplicity,
and suggest that history drives us to err on the side of simplicity.

\begin{recommendation}
    {rec:comp:debug:use}
    {Debugging mode shoud be binary: it is either enabled or disabled, with no further configuration}
    For example, within a \texttt{Task}, one might check if debugging mode is
    enabled with a statement like \texttt{if self.config.debug: ...}.
\end{recommendation}

\begin{note}
As of 2018-08-21, I (JDS) have removed references to enabling debugging
through the task config system. This is based on OOB discussions that the
config is most appropriate for recording options which affect the science
outputs, whereas (one assumes) that the code should generate the same science
results whether or not debugging is enabled. I think that means it really
wants to be driven by a command line argument and a set of explicit arguments,
but I am happy to be persuaded otherwise.
\end{note}

The natural consequence of this is that debugging of tasks involved by some
form of middleware (e.g. \texttt{CmdLineTask}, \texttt{PipelineTask} called by
an executor) should receive an explicit argument enabling debugging. All other
callables which support debugging must take an explict argument which enables
it.

\subsection{Logging}
\label{sec:comp:log}

This component is derived from \S\ref{sec:design:debug}.

Logging is an important aspect of running large data processing.  It is also
integral to quality assessment as the logging information provides important
contextual information when inspecting data for quality issues.  Specifically,
log messages presenting information about the processing: e.g. PSF width,
number of stars used in a model fit, can indicate problems with the
algorithmic behavior or input data.  Logging also provides information about
potential causes for unexpected termination including exit codes and
exceptions.

DM already provides a logging system (the ``log'' package), and the API
documentation provided for
it\footnote{\url{https://developer.lsst.io/stack/logging.html}} is
adequate\footnote{It is, perhaps, worth noting that configuring the output of the
logging system is much less straightforward and is badly documented:
\url{http://doxygen.lsst.codes/stack/doxygen/x_masterDoxyDoc/log.html\#configuration}.}.

However, the outputs of the logging system become increasingly hard to parse
as the number of concurrent processes increases: a common complaint when
running large-scale data processing is that it's difficult to identify
failures and understand where they came from.

\begin{recommendation}
  {rec:comp:log}
  {A log aggregation and monitoring service should be provided for large-scale processing jobs at the Data Facility}
Such a service should not be a requirement for jobs to execute (in particular,
when running in non-Data Facility environments, logs should continue to be
generated and collected as at present).
\end{recommendation}

Log aggregation should provide the following capabilities:

\begin{itemize}
\item{Display all log messages from a given pipeline execution;}
\item{Display messages at a selected log level (\texttt{INFO}, \texttt{WARN}, \texttt{ERROR}, etc);}
\item{Display time-ordered logs for a given thread;}
\item{Display logs, exit status and/or exceptions for threads ending in an unexpected state;}
\item{Searchable per process for regexp and timestamp.}
\end{itemize}

We believe that these capabilities could be provided by building atop
off-the-shelf log aggregation tools such as
Logstash\footnote{\url{https://www.elastic.co/products/logstash}}. However, we
suggest that exploring synergies with existing or planned Data Facility
services is likely essential. We emphasize that effective log management
capabilities are of increasing importance not just in the operational era but
in the immediate future, as we move towards large-scale data processing in
support of commissioning: deploying a basic version of this service should be
regarded as a high priority.

\subsection{Workload Management System}
\label{sec:comp:workload}

Derived from \S\ref{sec:design:debug}.

A commonly heard complaint from developers is that the logistics of running code at scale for test purposes is too complex and unreliable.
In particular, developers are concerned that sometimes jobs fail to execute without a reason being clearly stated (the job simply disappears from the Slurm queue, without explanation) or that it can be hard to understand from the logs why a job failed or particular output was generated.

The WG reached the conclusion that a wholesale rethinking of workload management is outside the scope of the group's charge.
Instead, we suggest just three key improvements: to logging (to better identify and diagnose failures), to provenance (to better understand why certain data products have been produced) and to documentation (primarily, to avoid confusion over why certain jobs might vanish without trace). Logging and provenance are addressed in \S\ref{sec:comp:log} and \S\ref{sec:comp:provenance} respectively.

\begin{recommendation}
  {rec:comp:workload}
  {Tutorial and reference documentation for developers attempting to run jobs at scale should be refreshed}
  In particular, revised documentation should focus on identifying and resolving common failure modes.
\end{recommendation}

\subsection{Visualization}
\label{sec:comp:vis}

Derived from \S\ref{sec:design:debug}.

Broadly, we regard ``visualization'' as an umbrella term covering both visualizations derived from catalogs as well as image display and manipulation.
The concerns expressed by developers and others are, on the whole, common to both; the solutions may not be.

For both visualization regimes, the predominant request from developers is that the project provide them with clear guidance as to both what resources will be provided and supported by the LSST construction effort (for example, tools like Firefly or abstractions like \texttt{afwDisplay}) and which tools they are required or expected to use in the interest of maintaining a coherent and consistent codebase and set of outputs.

In this section, we concentrate on ad-hoc visualization in support of the regular pipeline developer.
In \S\S\ref{sec:design:drill} and \ref{sec:comp:drill}, we describe the design of an interactive ``drill-down'' tool, which will provide a number of plotting and data exploration capabilities.
It is our expectation that, however comprehensive such a tool might become, there will always be a necessity for individual developers to be able to quickly investigate their data in as flexible a way as possible; conversely, wherever practical we should enable developers to exploit, and encourage them to remain consistent with, the capabilities and conventions delivered by the drill-down tool.
These sections should therefore be regarded as complementary.

\subsubsection{Catalog visualization}
\label{sec:comp:vis:catalog}

The group notes that there are many contexts in which visualizations derived from catalogs might be required (for example, in-line display in a Jupyter notebook, persisting plots from a debugging session, as described in \S\ref{sec:comp:debug}, preparing plots for publication, etc), which may all have substantially different requirements.
We also note that there exists a diverse infrastructure of scientific plotting and data exploration tools in the Python community, a comprehensive selection of which has been collated by the PyViz project\footnote{\url{http://pyviz.org}}, which also provides documentation on effectively using them in conjunction with each other.
Given that, we regard it as unnecessary, and indeed undesirable, for LSST to attempt to standardize on any particular plotting tool: we should rather encourage developers to exploit community resources with minimal overhead.

\begin{recommendation}
  {rec:comp:vis:catalog:pyviz}
  {DM should formally adopt the PyViz ecosystem}
This adoption would include, for example, including PyViz tools in a regular installation of the LSST Stack; providing training and documentation for developers and --- crucially --- developing interfaces which enable LSST conventions (afw tables, the Data Butler) to be used in the PyViz context.
\end{recommendation}

Many visualization use cases involve manipulating data at a larger scale than can conveniently be done on a single compute node.
Within the PyViz ecosystem, Dask\footnote{http://dask.pydata.org/en/latest/} is the preferred approach.
We note that there many be some redundancy between Dask and the LSST middleware (Butler, PipelineTask and appropriate executors, etc).
However, Dask provides a convenient, easy to install and use solution which can immediately address developer needs.

\begin{recommendation}
  {rec:comp:vis:catalog:dask}
  {DM should adopt Dask to enable users to work with larger than memory data}
This might be achieved by providing users with the ability to spin up Dask clusters on demand using (say) Kubernetes, or by providing a Dask cluster at the LSST Data Facility to which users can connect.
If ongoing middleware development renders this obsolete, then it can be retired.
\end{recommendation}

\subsubsection{Image visualization}
\label{sec:comp:vis:image}

\subsection{Provenance}
\label{sec:comp:provenance}

Derived from \S\S\ref{sec:design:debug} \& \ref{sec:design:drill}.

An effective provenance system is key to any form of QA work: it is clearly necessary to understand where a particular result came from in order to investigate issues it raises.
This is necssary for stand-alone quality analysis, but is fundamental to the proper operation of drill-down and metric-tracking systems.
Furthermore, some high level aggregate data products that are derived from provenance data --- for example, the number of images that contribute to each coadd patch --- are important in QA work.

The QAWG notes that provenance has long been discussed within DM, but detailed plans and timelines have historically been fragmentary\footnote{We hope the ongoing middleware development effort is changing this!}.
We are concerned that the lack of an effective provenance system is a major barrier to productivity.

\begin{recommendation}
  {rec:comp:provenance}
  {The design and implementation of the provenance system should have
  high priority in the project scheduling.}
\end{recommendation}

The QAWG believes that the requirements on provenance tracking are adequate as described in the Data Management Middleware Requirements (\citeds{LDM-556}) and Data Management Data Backbone Services Requirements (\citeds{LDM-635}).
QA use cases provide no further requirements.

In the Generation 3 Middleware design, Butler/PipelineTask framework is responsible of producing provenance.
For production runs, the Data Backbone Services may collect and store additional provenance information --- for example from the Batch Production Service --- in addition to that generated by Butler and/or PipelineTask.
In the QA use cases, though, we expect the primary source of provenance will be Butler and PipelineTask.

In the Generation 3 Middleware, each dataset can be linked to provenance information such as input datasets, pipeline definition, configurations, and software version (e.g. DMS-MWST-REQ-0024 and DMS-MWBT-REQ-0096 in LDM-556).
Assuming that the recommendation of \S\ref{rec:metric_dataId} is adopted, \glspl{metric value} will be stored as Butler datasets and will have associated Data IDs.
This will enable their provenance to be traced in the same way as other datasets.

The QAWG does not place strong requirements on per-source provenance of catalogs and database records.
The current design is that each database record in the production database is either ingested from a file, of which the full provenance is traced, or from an uniquely identifiable execution.
This design does not directly provide detailed per-source provenance --- such as which input images acually contribute to the measurement of a particular source --- but rather enables us to trace the full set inputs that \textit{could have} contributed to the source.
We suggest that this is adequate; if necessary, additional tooling to e.g. investigate the composition of coadds can be added to the drill-down system (\S\ref{sec:design:drill}).

\subsection{Code quality documentation}
\label{sec:comp:code_quality}

Derived from \S\ref{sec:design:test}.

DM values code quality, with an elaborate set of coding standards\footnote{\url{https://developer.lsst.io/coding/intro.html}}, substantial unit test suites, regular code review, and a set of automatic tests for compliance with code style rules.
However, while essential, these are often inconsistently applied, and developers are left confused about what is actually required of them.

\subsubsection{Unit tests}
\label{sec:comp:code_quality:unittest}

The scope of DM's unit test system is not well defined.
Tests range from true unit tests---limited in scope to one ``unit'' of code---to what are effectively integration tests, relying on functionality from many packages working in concert and testing that the results meet some (often apparently arbirary) numerical threshold.
At this stage in the construction project, we do not believe that a wholesale attempt to refactor or reconsider the way that these tests are constructed is plausible, though: we suggest that the current situation is tolerable for the remainder of the project.

However, developers continue to take inconsistent approaches to testing, and disagree (occasionally publicly) about what it is necessary to test, in how much detail, and in what way.
We further note that the Developer Guide provides advice which is obsolete and widely ignored\footnote{\url{https://developer.lsst.io/coding/unit-test-policy.html}}.
We believe that refreshing the developer-facing documentation, together with closer attention to the form and structure of tests in code reviews (\S\ref{sec:comp:code_quality:review}), will pay dividends in terms of reducing confusion and the potential for disagreement.

\begin{recommendation}
    {rec:comp:code_quality:unittest}
    {Obsolete and unclear sections of the Developer Guide should be rewritten to provide clearer guidance on unit tests}
    This should include, for example:
\begin{itemize}

  \item{Guidance for unit vs. integration tests, and which packages it is appropriate to write tests in (e.g. is it adequate for certain functionality only to be tested through packages like ci\_hsc?);}
  \item{Requirements for code coverage;}
  \item{Appropriate languages for writing tests (should C++ code be tested in C++, or is it acceptable---or encouraged---to test only the Python-wrapped version?;}
  \item{Are there certain types of code which it is appropriate not to test (e.g. boilerplate accessor methods)? How exhaustive should tests be?}

\end{itemize}

\end{recommendation}

\subsubsection{Code review}
\label{sec:comp:code_quality:review}

Various aspects of the unit test system, e.g. coverage requirements, are best enforced through code review.
However, we currently provide minimal written guidelines to developers about what code reviewers should be looking for\footnote{e.g. ``is there adequate unit test coverage for the code?'', but with no guidance on what constitutes ``adequate''}.

\begin{recommendation}
    {rec:comp:code_quality:review}
    {The Developer Guide should be expanded to provide checklist-style documentation for code reviewers making clear what is expected from them
during the review.}
\end{recommendation}



\subsection{Documentation and examples}
\label{sec:comp:doctest}

Derived from \S\ref{sec:design:test}.

Across the codebase there are scripts, example code, and other utilities.
Often these are lurking in \texttt{examples} directories in stack packages.
Occasionally they are exposed as Jupyter notebooks, sometimes living in separate repositories.
These examples are generally hard to discover; often, they are only made available to new team members (or external third parties) by chance conversation.

\begin{recommendation}
    {rec:sec:comp:doctest:index}
    {Provide a central location where examples, scripts and utilities which are not fundamental to pipeline execution are indexed and made discoverable}
See also \jira{DM-15807}.
\end{recommendation}

We observe that many of these examples are old, obsolete and often broken.
We note that users attempting to run these examples will frequently report that they are found to be non-functional.
We further observe that developers are unclear about their obligations for updating these examples when writing new code, an issue which is compounded when the code being changed is in a different package from the affected example.
Finally, we regard broken examples as contributing to an actively hostile user experience: no example is better than a misleading or failing one.

\begin{recommendation}
    {rec:sec:comp:doctest:policy}
    {The Project should adopt a documented (in the Developer Guide) policy on the maintenance of example code}
\end{recommendation}

Ultimately, \emph{all} code --- include examples --- should be tested in the \gls{ci} system: see below.
However, pending a mechanism for this, we suggest that:

\begin{itemize}
    \item{Developers are \textit{not} required to search the codebase for examples which may be affected by changes they are making elsewhere;}
    \item{
        When a broken example is discovered, it may be fixed if the changes required are trivial.
        However, if substantial effort would be needed, the example should simply be removed and an issue filed on Jira to request its reinstatement in future.
    }
\end{itemize}

Ultimately, we suggest that examples should be tightly intyegrated with the overall documentation effort.


\begin{recommendation}
    {rec:sec:comp:doctest:ci}
    {The Project should prioritise the development of a documentation system which makes it convenient to include code examples and which tests those examples as part of a documentation build}
\end{recommendation}

There are various technologies which could be adoped to to address this goal\footnote{For example, \href{https://jupyter.org/}{Jupyter notebooks} or \href{http://www.sphinx-doc.org/en/stable/ext/doctest.html}{Sphinx doctests}.}.
The WG suggests that standardizing upon a single technology is essential, but takes no position as to which technology is most appropriate.

Finally, we note that the the same concerns apply to executable code (in \texttt{bin} directories) which is not regularly tested in CI.

\subsection{CI system updates}
\label{sec:comp:ci}

Derived from \S\ref{sec:design:test}.

\assign{John}

\begin{itemize}

  \item{Test coverage.}
  \item{Tighter control of the environment.}
  \item{Better notifications.}
  \item{Better descriptions of which jobs do what.}
  \item{Clear description of what Developers are required to do before merging
  to master (see also \S\ref{sec:comp:code_quality}).}

\end{itemize}

\subsection{Metrics Dashboard / SQuaSH}

Derived from \S\ref{sec:design:test}.

\assign{Angelo}

To date, SQuaSH has been used to follow a subset of KPMs computed by validate\_drp for tracking performance regressions due to pipeline changes by regularly reprocessing test datasets in Jenkins/CI.

The following recommendations would enhance SQuaSH capabilities for DM developers.

\begin{recommendation}
SQuaSH should be used by developers for tracking metrics on their particular projects.
\end{recommendation}

Developers can instrument their science pipeline Tasks using \texttt{lsst.verify} and create new verification packages to be tracked in SQuaSH (see e.g. \texttt{jointcal}). It would be interesting to send results to SQuaSH when testing development branches, so that developers can compare the new metric values with the previous values \textit{before} merging to master. Any metric defined in \texttt{lsst.verify.metrics} should be uploaded to SQuaSH including, for example, computational metrics like code execution time.

\begin{recommendation}
SQuaSH should provide automated notification of regressions.
\end{recommendation}

Metric specifications in \texttt{lsst.verify} include thresholds that can be used to automatically detect and notify regressions. The notifications could be presented to developers by Slack, for example.

\begin{recommendation}
SQuaSH should provide a metric summary display.
\end{recommendation}

Verification packages might have specialized visualizations for displaying metric summary information in addition to the current time series plot. DM developers should be able to extend SQuaSH by creating new visualizations following developer documentation provided in the SQuaSH Documentation (https://squash.lsst.io/)

\begin{recommendation}
SQuaSH should support the LDF execution environment in addition to Jenkins/CI.
\end{recommendation}

Pipeline runs on larger datasets (e.g. HSC RC2 weekly reprocessing) require more computation than can be provided in the Jenkins/CI environment. SQuaSH should be flexible to support other environments like the LDF environment.

\begin{recommendation}
SQuaSH should be able to store and display metric values per DataIds (e.g. CCD, visit, patch, tract, filter).
\end{recommendation}

Pipeline runs on larger datasets (e.g. HSC RC2 weekly reprocessing) also require to store and display metric values per \texttt{DataIds} as opposed to the entire dataset (e.g. test datasets in Jenkins/CI). The ability to identify metric values per filter name or spatially by CCD in a visit or per patch in a tract, would enhance SQuaSH display and monitoring capabilities, turning SQuaSH or its successor into a richer metric dashboard (see also \S\ref{sec:design:drill_down}).

\subsection{Metric Collection Framework}
\label{sec:comp:verify}

...something about lsst.verify, design reviews, capabilities, pipeline
integration, DMTN-057, etc.

\subsection{Standard format dataset packages}
\label{sec:comp:dataset}

Derived from \S\ref{sec:design:test}.

The DM subsystem curates a number of ``datasets'': collections of data related by some underlying theme or use case.
These themes might include originating from the same instrument or facility (e.g. testdata\_cfht); being used to test a particular package (e.g. testdata\_jointcal); or addressing some particular science case (e.g. ap\_verify\_hits2015).

Currently, DM's datasets are heterogeneous: there is no accepted standard for the type of data that a dataset should contain, and nor is there any standard for where the dataset is stored or how it is curated.
Some of DM's datasets are stored on \gls{gpfs} at the LSST Data Facility; some are made available through Git LFS\footnote{\url{https://git-lfs.github.com}; \url{https://developer.lsst.io/git/git-lfs.html}}; some contain only raw data; some contain calibration data; some contain processed data; some are regularly updated; some have documentation describing in detail what the dataset contains.

This lack of standardization limits the reuse of datasets (it is hard to reuse a dataset curated for one purpose for some other test unless one fully understands its contents) and means that developers often struggle to find the most appropriate data to test or debug some particular algorithm.

\begin{recommendation}
    {rec:comp:dataset:standard}
    {A standardized format for dataset repositories should be adopted across DM}
Obviously, not all repositories will have exactly the same contents: in some cases, it may be necessary for a repository to contain (say) calibration products, while in others they may be inappropriate.
However, it should be possible to establish at a glance what the contents of each dataset is; if calibration products \emph{are} included, it should be immediately obvious what they are and how to apply them.
\end{recommendation}

The key desiderata for standardizing the format of test datasets are:

\begin{itemize}
\item{
    Developers would like low-friction access to test datasets.
    A central location where developers can look up what has been curated is desired.
}
\item{
    Developers would like datasets at multiple scales and representative of various data quality and observing conditions.
    The properties of these datasets must be well understood.
}
\item{
    Besides raw (unprocessed) data, intermediate and final data products from various stages of pipeline processing are desired.
    They facilitate testing of algorithms which are only relevant to later parts of the pipeline or analysis codes without the need of regenerating the products.
}
\item{
    Datasets require continuing maintenance.
    Data products based on a recent software release are usually wanted (although older ones may also be useful to test backwards compatibility).
}
\end{itemize}

These considerations aside, this WG does not make a specific recommendation about the detailed layout of datasets, beyond noting the existence of the Common Dataset Organization and Policy \footnote{\url{https://developer.lsst.io/services/datasets.html}} which might form a convenient basis for further development.

Regardless of the layout adopted, we expect that the dataset will need to evolve with time: as new versions of the LSST code are released, expectations about both the contents of data repositories and the format of persisted data products will change, occasionally in backwards-incompatible ways.
For a dataset to remain useful, active curation is required.

\begin{recommendation}
    {rec:comp:dataset:owner}
    {Each dataset should have an explicitly named product owner}
Product owners are responsible for ensuring that the content and use cases of the datasets are well described and compatible with recent stack versions.
The owner of a dataset could often be a member of the team with immediate use cases and knowledge of the relevant camera package.
\end{recommendation}

The variety of different sources and use-cases for datasets means that they span a wide range of sizes.
It is therefore impractical to store and distribute them all in the same way.
Instead, the QAWG suggests they can usefully be divided into two categories: \emph{small} and \emph{large}.

\begin{description}

\item[Small] datasets are those smaller than 100\,GB in total size.
They are intended for use, for example:
\begin{itemize_single}
    \item{as input test data in \gls{ci} jobs;}
    \item{as example data in documentation, demos, and tutorials.}
\end{itemize_single}
We recommend that they are:
\begin{itemize_single}
    \item{packaged as EUPS products;}
    \item{made publicly available on GitHub;}
    \item{stored as Git LFS repositories as needed;}
    \item{versioned with DM software releases.}
\end{itemize_single}

\item[Large] datasets are bigger than 100\,GB in total size.
They are intended for use, for example:
\begin{itemize_single}
    \item{in large scale integration tests;}
    \item{in flushing out edge cases which may not be apparent from smaller data volumes;}
    \item{for archiving outputs from engineering facilities like the Camera test stands.}
\end{itemize_single}
We recommend that they are:
\begin{itemize_single}
    \item{made available on project-provided shared network filesystems (i.e. \gls{gpfs});}
    \item{protected under a disaster recovery policy.}
\end{itemize_single}

\end{description}

\begin{recommendation}
    {rec:comp:dataset:storage}
    {Datasets may be stored on either shared filesystems or Git LFS as appropriate, depending on the total size of the dataset}
\end{recommendation}

\subsection{Standard test package design}
\label{sec:comp:test_pkg}

Derived from \S\ref{sec:design:test}.

\assign{Hsin-Fang}

Currently, automatic continuous integration tests are performed via
multipla packages under two designs: (1) Scons-based execution, including
ci\_hsc and ci\_ctio0m9, and (2) exeuction through shell scripts
in validate\_drp.  Both ci\_hsc and validate\_drp are run in
Jenkins and triggered by timers every night (\S\ref{sec:comp:ci}).

It's QAWG's understanding that the validate\_drp scripts will
eventually replace ci\_hsc and ci\_ctio0m9, and a set of test scripts
will be run in a meta-package named lsst\_ci.  However, at the time
of writing, the validate\_drp scripts test only the single frame
processing step, while ci\_hsc exercises almost the entire end-to-end
DRP pipelines.  There has not been sufficient resources in implementing
further processing in validate\_drp.  Similarly, the lsst\_dm\_stack\_demo
repository should be converted into an EUPS product and a test
script added to lsst\_ci for execution (\jira{DM-14806}).

The QAWG recommends priority to unify the CI test package design and finish
the transition to validate\_drp. If such effort cannot be allocated,
documentations should be added to clearly describe the status quo, and
recommendations for developers during the transition should be added to the
Developer Guide (similar to \S\ref{sec:comp:code_quality} and
\S\ref{sec:comp:ci}). Before validate\_drp can replace ci\_hsc and
ci\_ctio0m9, the packages should be maintained.


\begin{recommendation}
  {rec:comp:test_pkg:ci}
  {The CI design should address the use cases of existing packages
   (lsst\_dm\_stack\_demo, ci\_hsc, validate\_drp) and unify them into
   one consistent design}
\end{recommendation}
\begin{recommendation}
  {rec:comp:test_pkg:doc}
  {The current status quo of the CI system should be well documented
   and updated during the transition}
\end{recommendation}

\subsection{Drill-down workflows}
\label{sec:comp:drill}

Currently, pipeline developers rely on a variety of ad-hoc visualization tools and custom workflows to debug processing problems and investigate the effects of new algorithms.
The QAWG recommends development of a browser-based dashboard QA system that is designed to cater to the workflow of the debugging developer, and is informed and guided by current usage of existing tools (e.g. \texttt{pipe\_analysis} and \texttt{qa\_explorer}).
This is separate from, and in addition to, the SQuaSH dashboard (\S \ref{sec:comp:metric:dashboard}).

\begin{recommendation}
  {rec:comp:drill:dashboard}
  {DM should develop a browser-based interactive dashboard that can run on any pipeline output repository (or comparison of two repositories) to quickly diagnose the quality of the data processing}
  This dashboard should have two levels of detail: a high-level summary of top-level global metrics (Fig. \ref{fig:comp:drill:quick}), and and a metric view showing more information on a selected metric (or set of metrics; Fig. \ref{fig:comp:drill:metric}).
  The more detailed metric dashboard should be able to explore both coadds and individual visits.
\end{recommendation}

\begin{figure}
  \begin{center}
    \includegraphics[width=0.8\textwidth]{figures/quickLook.png}
  \end{center}
  \caption{
    Mock-up of the ``quick look'' overview screen of the drill-down system.
    This provides a summary of all metrics calculated over the results of a particular pipeline execution, indicating any that fell below threshold.
  }
  \label{fig:comp:drill:quick}
\end{figure}

A developer will load the dashboard by visiting a particular URL.
They will then enter the path to a data repository\footnote{On some accessible filesystem; this assumes that the QA system is running e.g. on a host at the LSST Data Facility with access to \texttt{/scratch} or \texttt{/project}} which contains metric values.
They will be presented with the ``quick look'' screen as shown in Fig. \ref{fig:comp:drill:quick} which summarizes the results of the data processing.

The dashboard will generate summary information on the fly by introspecting the repository.
It follows that, in addition to the metric values themselves, the repository will also contain metadata that describes:
\begin{itemize}
    \item{Which metrics were computed?}
    \item{What selection was done on the source catalog to compute each metric?}
    \item{What is an acceptable value for each metric for this particular data set?}
\end{itemize}
The developer will have a choice to explore either the metrics in a single repository, or the differences in metrics between two data repositories by entering a ``comparison repository'' in addition to the primary one (that necessarily would need to have the same metrics computed as the primary).

The high-level landing page of this dashboard should allow for at-a-glance summary of the data and identification of any problems.
The overall data summary could be in a header displaying the some numbers of interest, such as number of tracts, numbers of visits per filter, total numbers of sources, etc.
At-a-glance identification of problems could be accomplished by displaying fully-rolled-up metric summary values in a minimalistic but informative format, such as an array of color-coded buttons or a color-coded table (e.g., green for good, red for bad).
This top-level dashboard page should also have minimal but useful visualizations, such as an RA/dec plot of a single metric value (perhaps switchable via drop-down menu).

\begin{figure}
  \begin{center}
    \includegraphics[width=0.8\textwidth]{figures/metricView.png}
  \end{center}
  \caption{
    Mock-up of the ``metric view'' showing more information on how a particular metric value was calculated.
  }
  \label{fig:comp:drill:metric}
\end{figure}

Selecting any metric from the top-level page should load up a metric-detail page, as shown in Fig. \ref{fig:comp:drill:metric}, which should allow for more detailed exploration of an individual metric.
The layout and function of this page will depend on the type of metric.
As an example, for metrics derived directly from individual source measurements, it might display a scatter plot of the metric values vs. source magnitude, as well as an RA/Dec scatter plot colormapped by metric value.
This could by default show the data for all tracts, but tract-aggregated information could be available for each tract on hovering over the sky map.
Upon clicking upon a specific tract, then only the points for that tract will be selected (both in sky plot and scatter plot), and then the sky plot would then display patch-summarized data.
Similarly, from here, clicking on a patch would load up just the data in that patch.

This metric dashboard should also allow simultaneous visualization of different metrics, to allow cross-filtering.
This might be accomplished, for example, by having a sidebar listing the metrics, and being able to use it to either switch between metrics or to select multiple metrics.

There should be an analogous drill-down mode for exploring visit-level data, with the drill-down levels being visit$\rightarrow$CCD rather than tract$\rightarrow$patch.
There should be seamless integration between coadd and visit mode, to match the typical workflow of a debugging developer, who will first see that there is a problem with a particular metric in a particular tract at the coadd processing level, and then will want to see what visits went into that coadd tract.
To enable this, from the ``coadd mode'' metric dashboard, there could be a way to toggle on/off visit outlines, and to jump to ``visit mode'' for a selected visit.
In ``visit mode,'' in addition to the scatter/sky plots that ``coadd mode'' has, there could be an additional figure showing the aggregated metric value as a function of visit number, where outlier visits would be clearly visible.
This system would enable a developer to quickly identify bad visits for a particular metric, in just a few clicks.

\begin{recommendation}
    {rec:dashboard:jupyter}
    {The dashboard should enable the analyst to start a Jupyter notebook session with the relevant datasets already loaded}
\end{recommendation}

From the metric dashboard, there could be an ``explore in Jupyter'' button that would load up the dataset in the JupyterLab environment, which would provide all the tools to make the dashboard visualizations, with the additional flexibility for ad-hoc exploration that the notebook provides.



\section{Conclusion}

This document has described the deliberations and conclusions of the QA Working Group.
It has taken a wide-ranging view over various aspects of the DM Subsystem, and presented a wide range of recommendations, which are summarised in Appendix \ref{sec:recs}.
Many of these recommendations are evolutionary improvements to existing DM tools, practices or documentation.
A few involve the development of new capabilities.
Of particular note in this latter capability are the call for the development of a integrated drill-down system, described in \S\ref{sec:comp:drill}, and for adoption of the Dask system \S\ref{sec:comp:vis}\footnote{We note that, at time of writing, some work involving Dask is now underway within DM, although we are not aware of design documentation describing exactly what capabilities will be provided.}.
These capabilities will require significant resources to deliver, and will therefore require action by DM Project Management.
However, we also commend to management some of the lower-profile recommendations: in particular, we feel that modest improvements to dataset organization and to the \gls{ci} system could have major impacts on DM's overall productivity.

\appendix
\printrecs
\glsaddall
\renewcommand*{\glsautoprefix}{glo:}
\printglossary[style=index,numberedsection=autolabel]

\bibliography{lsst,lsst-dm,refs_ads,refs,books}

\end{document}
