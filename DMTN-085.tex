\documentclass[DM,authoryear,toc,lsstdraft]{lsstdoc}
\usepackage[nonumberlist,nogroupskip,toc,numberedsection=autolabel]{glossaries}
\makeglossaries
%
% Acronyms
%

\newacronym{ci}{CI}{Continuous Integration}
\newacronym{dm}{DM}{Data Management}
\newacronym{hsc}{HSC}{Hyper Suprime-Cam}
\newacronym{kpm}{KPM}{Key Performance Metric}
\newacronym{sdqa}{SDQA}{Science Data Quality Assurance}
\newacronym{qa}{QA}{Quality Assurance}
\newacronym{qawg}{QAWG}{QA Strategy Working Group}

%
% Terminology
%

\newglossaryentry{aggregation}
{
  name={aggregation},
  description={A single result---e.g., a \gls{metric value}---computed from a
  collection of input values. For example, we can sum or average a
  \gls{metric} computed over patches to produce an \gls{aggregate metric} at
  tract level}
}

\newglossaryentry{aggregate metric}
{
  name={aggregate metric},
  description={An \gls{aggregation} of multiple \glspl{point metric}. For
  example, the overall photometric repeatability for a particular tract given
  multiple observations of each star}
}

\newglossaryentry{dashboard}
{
  name={dashboard},
  description={A visual display of the most important information needed to
  achieve one or more objectives, consolidated and arranged on a single screen
  so that the information can be monitored at a glance \citep{Few:2013}}
}

\newglossaryentry{drill down}
{
  name={drill down},
  description={Move from a higher level aggregation of data to its inputs. For
  example, given data describing a tract, we might drill down to constituent
  patches and then to objects; given a visit, we might drill down to CCD and
  then source. In the context of this document, it refers to the act of
  identifying an issue in a high-level summary of the data (e.g. an aberrant
  \gls{metric value}) and interactively investigating its inputs to find the
  source of the problem}
}

\newglossaryentry{gpfs}
{
  name={GPFS},
  description={IBM's General Parallel File System; now known as IBM Spectrum
  Scale. In DM use, this is taken to mean bulk data storage provided through a
  POSIX filesystem interface at the LSST Data Facility}
}

\newglossaryentry{metric}
{
  name={metric},
  description={We follow the \citeds{SQR-019} definition of a metric as a
  measurable quantities which may be tracked. A metric has a name,
  description, unit, references, and tags (which are used for grouping). A
  metric is a scalar by definition. We consider multiple types of metric in
  this document; see \gls{aggregate metric}, \gls{model metric}, \gls{point
  metric}}
}

\newglossaryentry{metric value}
{
  name={metric value},
  description={The result of computing a particular \gls{metric} on some given
  data. Note that we \textit{compute}, rather than measure, metric values}
}

\newglossaryentry{model metric}
{
  name={model metric},
  description={A \gls{metric} describing a model related to the data. For
  example, the coefficients of a 2D polynomial fit to the background of a
  single CCD exposure}
}

\newglossaryentry{monitoring}
{
  name={monitoring},
  description={The process of collecting, storing, aggregating and visualizing
  metrics}
}

\newglossaryentry{point metric}
{
  name={point metric},
  description={A \gls{metric} that is associated with a single entry in a
  catalog. Examples include the shape of a source, the standard deviation of
  the flux of an object detected on a coadd, the flux of an source detected on
  a difference image}
}

\newglossaryentry{releaseable product}
{
  name={releaseable product},
  description={A software package or other component of the DM system which
  is expected to be included in the next tagged release of the system. At time
  of writing, this implies inclusion in a standard top-level package
  (e.g. lsst\_distrib), but we note that future changes to the release procedure
  may render that definition obsolete}
}

\newglossaryentry{squash}
{
  name={SQuaSH},
  description={Science Quality Analysis Harness; \citeds{SQR-009};
  \url{https://squash.lsst.codes}}
}

\newglossaryentry{tidy data}
{
  name={tidy data},
  description={Tidy datasets have a specific structure: each variable is a
  column, each observation is a row, and each type of observational unit is a
  table \citep{JSSv059i10}}
}


\input meta.tex

\title{QA Strategy Working Group Report}

\author{%
Bellm, E.C.,
Chiang, H.-F.,
Fausti, A.,
Krughoff, K.S.,
MacArthur, L.A.,
Morton, T.D.,
Swinbank, J.D and
Roby, T.
}

\setDocRef{\lsstDocType-\lsstDocNum}
\date{\vcsDate}

\setDocAbstract{%
Abstract.
}

\setDocChangeRecord{%
  \addtohist{\vcsRevision}{\vcsDate}{Unreleased draft.}{Bellm et al.}
}

% For specific WG recommendations
\newenvironment{recommendation}[1][]
{
  \begin{admonition}{green!5!white}{green!75!black}{Recommendation}{#1}
}
{
  \end{admonition}
}

\begin{document}

% For tracking who's doing what
\newcommand{\assign}[1]{
  \begin{admonition}{red!5!white}{red!75!black}{Assignee}{}#1\end{admonition}
}

\maketitle

\section{Introduction}
\label{sec:intro}

\assign{John}

This report constitutes the primary artefact produced by the DM \gls{qawg},
addressing its charge as defined in \citeds{LDM-622}.

The complete scope of ``\gls{qa} within \gls{dm}'' is too large to be
coherently addressed by any group on a limited timescale. Per its charge,
then, the \gls{qawg} has constrained itself to considering only those aspects
of \gls{qa} which are most directly relevant to the construction of the LSST
Science Pipelines. In particular, we have considered the tools which
developers need to construct and debug individual algorithms; tools which can
be used to investigate the execution of at-scale pipeline runs; and tools
which can be used to demonstrate that the overall system meets its
requirements (to ``verify'' the system). This deliberately excludes broader
requirements of Commissioning, Science Validation, or run-time
\gls{sdqa}\footnote{Effectively, code executed during prompt or data release
production processing to demonstrate that the data being both ingested and
released is of adequate quality.}.

This report consists broadly of three parts. In \S\ref{sec:approach}, we
describe the approach that the \gls{qawg} took to addressing its charge. In
\S\ref{sec:design}, we present a high-level overview of the systems that we
envisage the future DM comprising. Finally, In \S\ref{sec:comp} we identify
specific components --- which may be pieces of software, documentation,
procedures, or other artifacts --- that should be developed to enable the
capabilities we regard as necessary. In some cases, these components may be
entirely new developments; in others, existing tools developed by the \gls{dm}
subsystem may already be fit for purpose, or can be adapted with some effort.
We have noted when this is the case.

Finall,y in Appendix \ref{glo:main} we define a number of key terms which are
used throughout this report and which we we suggest be adopted throughout
\gls{dm} to provide an unambigous vocabulary for referring to QA topics.

\section{Approach to the Problem}
\label{sec:approach}

\assign{John}

\subsection{Pipeline debugging}

\assign{John}

What tools do we need to help pipeline developers with their everyday work?

\begin{itemize}

  \item{How do you go about debugging a \texttt{Task} that is crashing?}
  \item{Is \texttt{lsstDebug} adequate?}
  \item{Do we need an afwFigure, for generating plots, to go alongside \texttt{afwDisplay}, for showing images?}
  \item{What additional capabilities are needed for developers running and debugging at scale, e.g. log collection, identification of failed jobs, etc.}
  \item{What's needed from an image viewer for pipeline developers? Is DS9 or Firefly adequate? Is there value to the afwDisplay abstraction layer, or does it simply make it harder for us to use Firefly's advanced features?}
  \item{How do we view images which don't fit in memory on a single node?}
  \item{How do we handle fake sources? Is this a provenance issue?}

\end{itemize}

\subsection{Drill down}

\assign{John}

\subsection{Datasets and test infrastructure}

\assign{John}

\section{Design sketch}
\label{sec:design}

In this section, we summarize the issues identified and approaches suggested
by the groups described in \S\ref{sec:approach}. From these, we synethesize a
number of concrete actions --- tools to be developed, documentation to be
provided, etc --- we which recommend to the project in \S\ref{sec:comp}.

\subsection{Pipeline debugging}
\label{sec:design:debug}

\assign{John}

The group does not identify a single, over-arching tool or concept which would
solve the problem of developer productivity and happiness. Instead, we believe
that developer needs can be best addressed by making incremental improvements
to a number of core pieces of \gls{dm} technology and infrastructure which
team members regularly interact with. In particular, we identified the
following major ``pain points'' for developers:

\begin{itemize}

  \item{The pipeline debugging system, lsstDebug, is badly documented and
  awkward to use, and developers lack appropriate guidelines on how to use it
  effectively.}

  \item{Developers find it hard to know how to debug their code when it is
  running at scale. Issues include parsing logs when a large number of
  concurrent processes are running; inadequate documentation of the existing
  Slurm system; uncertainty about replacement of Slurm with a future workflow
  system; and difficulties in understanding the provenance of data products.}

  \item{The project has issued unclear guidance and inadequate documentation
  to developers about the appropriate tools to be used for visualizing data.
  This is most pronounced for image visualization\footnote{Developers have the
  impression they ``ought to'' be using Firefly, but there is much uncertainty
  around its suitability for the task and its future development direction.},
  but also applies to catalog data.}

  \item{Developers struggle to get access to data for running tests --- both
  small and large scale --- in a convenient form. Large repositories exist on
  the project \gls{gpfs} system, but it's unclear what they contain or how
  to effectively access them\footnote{A canonical example is the \gls{hsc}
  public data release: the volume of data is overwhelming for a developer who
  simply needs some representative \gls{hsc} data to test an algorithm.};
  smaller repositories exist on GitHub\footnote{e.g.
  \href{https://github.com/lsst/validation_data_hsc}{validation\_data\_hsc},
  \href{https://github.com/lsst/afwdata}{afwdata},
  \href{https://github.com/lsst/ap_verify_hits2015}{ap\_verify\_hits2015},
  etc.}, but they are inconsistent in structure, content and documentation:
  it's impossible for a developer to quickly identify data which is relevant
  to their use case, or to establish whether some particular reduction of the
  data is ``correct''. Instead, they rely on folklore and talking to peers to
  find data that ``worked for somebody else'', with (often) predictably
  frustrating results.}

\end{itemize}

To address these issues, the working group suggests the development of a
number of separate-but-related components. These include:

\begin{itemize}

  \item{An updated system for instrumenting running pipeline code;
  effectively, a revision of lsstDebug. This is developed further in
  \S\ref{sec:comp:debug}.}

  \item{A revised set of tooling for generating, aggregating and analyzing
  logs. This is developed further in \S\ref{sec:comp:log}.}

  \item{Revised documentation on interacting with the workload management
  system. This is developed further in \S\ref{sec:comp:workload}.}

  \item{Guidelines for the structure and maintenance of data repositories.
  This is developed further in \S\ref{sec:comp:dataset}.}

  \item{A clear roadmap for the development of visualization tools, and,
  derived from that, guidelines on how to apply those in the development of
  pipelines. This is developed further in \S\S\ref{sec:comp:image} \&
  \ref{sec:comp:vis}.}

\end{itemize}

In addition, the WG suggests that prioritization of developer-accessible
systems for tracking and understanding the provenance of pipeline outputs. For
additional comments on provenance from a QA perspective, refer to
\S\ref{sec:comp:provenance}

\subsection{Drill down}
\label{sec:design:drill}

Drill-down workflows center on the need to quickly and efficiently identify
data processing problems. Typically, these will be identified from
discrepancies identified at higher levels of summary and aggregation.

We therefore envisage a drill-down system which provides rapid retrieval of
relevant quantities (metric values, image cutouts, catalog overlays, etc.)
combined with readily-(re)configurable interactive plotting tools. This is
provided in a browser-based tool which enables the user to rapidly access
successive layers of detail on aggregated metrics (effectively
``de-aggregating'' them on demand), and ultimately enables the user to
seamlessly transition to an interactive analysis environment\footnote{i.e. a
Jupyter notebook} primed with the data under investagion. Further details
about the design and capabilities of such a system are presented in
\S\ref{sec:comp:drill}

We suggest that this system should be linked with a metric tracking system:
selected \glspl{aggregate metric} from successive comparable\footnote{i.e.
using the same input data, running with the same configuration} processing
should be tracked as a time series, with the user able to identify outliers
and rapidly switch to the drill-down system to investigate. This capability is
an extension of that already provided by \gls{squash}, and is discussed
further in \S\ref{sec:comp:squash}.

The system implementing these capabilities must be able to handle large
datasets --- for example, an entire HSC public data release --- quickly.
Furthermore, we emphasize the importance of ease-of-use for the QA analyst in
order to shorten debugging cycles.

\subsection{Datasets and test infrastructure}
\label{sec:design:test}

The group regards this part of its charge as qualitiatively different from the
other two (\S\S\ref{sec:design:debug} \& \ref{sec:design:drill}): where those
focus on servicing the needs of the individual developer or analyst, this
material describes the capabilities that are required by the overall subsystem
to track its progress and verify its deliverables.

We considered the following four aspects of this part of the charge:

\subsubsection{Small-scale unit and documentation tests}

The current unit test and continuous integration system functions well.
However, there are a number of improvements that we suggest to make it better.
These include clearer instructions to developers on the expectations for
testing (discussed in \S\ref{sec:comp:doc}) and assorted minor upgrades to the
\gls{ci} system, primarily to enable more convenient failure notifications and
tighter control of the test environment (\S\ref{sec:comp:ci}).

The single highest priority we identified was the lack of a current system for
performing CI on example code, documentation and (perhaps) Jupyter notebooks.
This has already rendered many of the examples provided in the current
codebase obsolete --- and because they aren't regularly tested, we have no way
to know what else is broken and will fail or otherwise confuse new (or
existing) users. We regard addressing this as one of the single highest
priorities for the project. It is discussed further in
\S\ref{sec:comp:doctest}.

\subsubsection{Integration tests}

DM's integration test needs are currently served by a heterogeneous mix of
approaches: from ``unit'' tests which effectively test the integration of
multiple packages, to explicit integration jobs executed by the CI
system --- of which there are multiple types, and no common
implementation standardf \footnote{See
\href{https://github.com/lsst/lsst_dm_stack_demo}{lsst\_dm\_stack\_demo},
\href{https://github.com/lsst/ci_hsc}{ci\_hsc},
\href{https://github.com/lsst/validate_drp}{validate\_drp}, etc.} --- to large
scale data processing exercises undertaken periodically at the
\gls{ldf}\footnote{e.g.
\url{https://confluence.lsstcorp.org/display/DM/Reprocessing+of+the+HSC+RC2+dataset}.}.

We posit that a unified and documented approach to integration testing will
provide lower implementation overheads, fewer surprised for developers, and
more predictable coverage for the project. We expand on the considerations and
design for such a system in \S\ref{sec:comp:test_pkg}.

Developing a standardized approach to integration testing requires a
standardized approach to the management of test datasets: this is addressed in
\S\ref{sec:comp:dataset}.

It also impacts upon the \gls{ci} system, and hence is relevant to
\S\ref{sec:comp:ci}.

Finally, we note that the way in which pipelines are defined and executed will
likely evolve rapidly over the remainder of the construction period as
technologies like SuperTask and Butler Generation 3 come into use. Given the
rapidly-moving nature of this work, the QAWG regards them as out of scope,
except insofar as we urge that QA tasks should be implemented and kept up to
date with these new frameworks as they become available.

\section{Core components}

\subsection{Updated pipeline debugging system}

\assign{Simon}

The debugging system must be both reasonably powerful and easy to use.
It should also be obvious from help/doc strings how to turn on debugging.
There is a tradeoff between granularity and usability.

The recommendation is that the debugging system be simplified to be configurable at the task level.
Debugging is turned on via a config parameter.
This allows for single sub-tasks to turn on debugging independently.
In practice this means that statements like \texttt{if lsstDebug:} would turn into \texttt{if self.config.debug}.

This brings up the obvious issue that free functions/non-task methods called in the \texttt{run} method of a task do not necessarily have the debugging flag passed into them.
It becomes the responsibility of the implementer to pass \texttt{self.config.debug} into methods that have debug functionality.

Derived from \S\ref{sec:design:debug}.

i.e. redesigned \texttt{lsstDebug}.

\subsection{Logging}
\label{sec:comp:log}

This component is derived from \S\ref{sec:design:debug}.

Logging is an important aspect of running large data processing.  It is also
integral to quality assessment as the logging information provides important
contextual information when inspecting data for quality issues.  Specifically,
log messages presenting information about the processing: e.g. PSF width,
number of stars used in a model fit, can indicate problems with the
algorithmic behavior or input data.  Logging also provides information about
potential causes for unexpected termination including exit codes and
exceptions.

DM already provides a logging system (the ``log'' package), and the API
documentation provided for
it\footnote{\url{https://developer.lsst.io/stack/logging.html}} is
adequate\footnote{It is, perhaps, worth noting that configuring the output of the
logging system is much less straightforward and is badly documented:
\url{http://doxygen.lsst.codes/stack/doxygen/x_masterDoxyDoc/log.html\#configuration}.}.

However, the outputs of the logging system become increasingly hard to parse
as the number of concurrent processes increases: a common complaint when
running large-scale data processing is that it's difficult to identify
failures and understand where they came from.

\begin{recommendation}
  {rec:comp:log}
  {A log aggregation and monitoring service should be provided for large-scale processing jobs at the Data Facility}
Such a service should not be a requirement for jobs to execute (in particular,
when running in non-Data Facility environments, logs should continue to be
generated and collected as at present.
\end{recommendation}

Log aggregation should provide the following capabilities:

\begin{itemize}
\item{Display all log messages from a given pipeline execution;}
\item{Display messages at a selected log level (\texttt{INFO}, \texttt{WARN}, \texttt{ERROR}, etc);}
\item{Display time-ordered logs for a given thread;}
\item{Display logs, exit status and/or exceptions for threads ending in an unexpected state;}
\item{Searchable per process for regexp and timestamp.}
\end{itemize}

We believe that these capabilities could be provided by building atop
off-the-shelf log aggregation tools such as
Logstash\footnote{\url{https://www.elastic.co/products/logstash}}. However, we
suggest that exploring synergies with existing or planned Data Facility
services is likely essential. We emphasize that effective log management
capabilities are of increasing importance not just in the operational era but
in the immediate future, as we move towards large-scale data processing in
support of commissioning: deploying a basic version of this service should be
regarded as a high priority.

\subsection{Capability for developers to run pipelines at scale}

\assign{Simon}

I believe we aggreed that if the logging system and provenance system are sufficient to meet recommendations, this aspect is essentially a solved problem.

I do think that it also depends on the orchestration layer, but I don't think that can be in the scope of this group.

Derived from \S\ref{sec:design:debug}.

\subsection{Guidance on visualization}
\label{sec:comp:vis}

\assign{Simon}
We decided that this section should first be seeded with contexts for visualization. We can then go through and give specific guidance.

Contexts:
\begin{itemize}
\item Notebooks in-line: matplotlib-ish
\item Dashboard like visualization: health status (grafana?)
\item Pipelines debugging visualization: both persisted PNGs and pop up vis
\item Purpose built QA visualization: Like the multisky plot from \texttt{qa\_explorer}
\item Exploratory: Single plots with interactivity, including zoom and pan
\item Linked plots: Interaction in one frame causes chnge of state in another plot
\item Image visualization: (covered elsewhere)
\item Publication plotting
\item Static plots for reports
\item Plotting from a non-notebook interactive setting
\item Rendering of cached time series: SQuaSH
\end{itemize}

Derived from \S\ref{sec:design:debug}.

We're requesting a set of guidelines for developers here, not a new framework
--- but that's still a concrete deliverable (it's just documentation, rather
than code). We might suggest that these guidelines be developed by a new WG,
per Simon's
suggestion\footnote{\url{https://confluence.lsstcorp.org/display/DM/Pipeline+Debugging+Design}}.

\subsection{Image viewer}

\assign{Trey}

Derived from \S\ref{sec:design:debug}.

As of 2018-06-12 we haven't converged on a solid recommendation here.

Key considerations:

\begin{itemize}

  \item{Firefly is the annointed solution being provided by DM to external
  stakeholders (commissioning, operations, etc). It feels right to everybody
  that we should be dogfooding it, and also benefitting from development being
  carried out for those stakeholders.}

  \item{Currently, Firefly is unappealing to developers (primarily, I think,
  because of slowness of user interface, and perhaps also due to installation
  issues). Can we resolve these issues?}

  \item{We'd want to support visualization in a number of different
  environments, for e.g.:

    \begin{itemize}

      \item{Inside a Jupyter notebook;}
      \item{As a standalone tool, \`a la DS9;}
      \item{Embedded in a \gls{dashboard}, \`a la JS9, Aladin-Lite, etc.}

    \end{itemize}
  }

  \item{Do we lose flexibility by mandating the use of a backend-agnostic API
  (\texttt{afwDisplay}) rather than going ``all-in'' on e.g. a custom Firefly
  interface?}

  \item{We'll need to do full focal plane visualization, which none(?) of the
  current tools support well.}

\end{itemize}

Options include:

\begin{itemize}

  \item{Do nothing; continue as we are, which means most people will use DS9
  and a few will drift to Firefly as commissioning ramps up.}

  \item{Issue some sort of edict that pipelines developers have to use
  Firefly.}

  \item{Encourage the use of some other tool (Ginga?) instead of or as well as
  some of the above.}

  \item{Probably others.}

\end{itemize}

Sounds like we need somebody from the QAWG to actually write some requirements
--- or a wishlist set of features we want --- here.


\subsection{Catalog visualization tools}

\assign{Lauren}

Derived from \S\ref{sec:design:debug}.

For visualizing bigger-than-memory catalogs. May include e.g. the capability
to spin up Dask clusters on demand, combined with
Holoviews/Datashader/whatever. Somebody who knows about this stuff needs to
write a summary...

\subsection{Provenance}

\assign{Hsin-Fang}

The QAWG recognizes the importance of provenance and the
implementation of the provenance system will impact QA work
significantly.  We recommend high priority to finalize the design
and implementation of the provenance system.

The QAWG believes that the requirements on provenance tracking are
adequate as described in the Data Management Middleware Requirements
(\citeds{LDM-556}) and Data Management Data Backbone Services
Requirements (\citeds{LDM-635}). QA use cases provide no further
requirements.

With the current design of the Gen 3 Middleware, each dataset
can be linked to provenance information such as input datasets,
pipeline definition, configurations, and software version
(e.g. DMS-MWST-REQ-0024 and DMS-MWBT-REQ-0096 in LDM-556).
QC metric values will be Butler datasets so they will be associated
with specific data IDs, and their provenance can be traced like
other datasets. In the Gen 3 Middleware design, Butler/SuperTask
framework is responsible of producing provenance.  For production
runs, the Data Backbone Services may store additional provenance,
for example from the Batch Production Services, besides the
Butler/SuperTask generated provenance.  In the QA use cases, the
primary source of provenance will be the Gen3 Butler/SuperTask
generated provenance.

Regarding provenance of the database records, the QAWG does not
place a strong requirement on per-source provenance.  The current
design is that each database record in the production database is
either ingested from a file, of which the full provenance is traced,
or from an uniquely identifiable execution.  This design does not
directly provide detailed per-source provenance, such as what exact
input images acually contribute to the measurement of a particular
source, but the full inputs that can contribute to the source.  The
provenance tracking of synthetic sources is also unclear.  The QAWG
thinks most low-level information can be uncovered through the drill
down system (\S\ref{sec:design:drill}).  Diagnostic information can
also be computed in the pipeline codes and stored as additional
columns.  We recommend the tooling to investigate the composition
of coadds be made in the drill down system and does not put this
requirement in the provenance system.

The QAWG notes that some high level aggregate data products that
are derived from provenance data can be useful in QA work.  For
example, the number of images that contribute to each coadd patch
can be obtained from provenance data.  However, the QAWG are not
immediately convinced that it's worth spending significant time
investigating what aggregate products need to be considered.

Derived from \S\ref{sec:design:debug}.

This section should note:

\begin{itemize}

  \item{That provenance is an immediate issue impacting QA work, so a solution
  is a priority;}

  \item{Some requirements as to the granularity at which provenance tracking
  is necessary for QA.}

\end{itemize}

\subsection{Documentation content updates}
\label{sec:comp:doc}

Derived from \S\ref{sec:design:test}.

\assign{John}

\begin{itemize}

  \item{Clearer guidance on unit tests.}
  \item{Clearer guidance on code review, with requirements for test coverage
  etc.}

\end{itemize}

\subsection{Testing for documentation}

Derived from \S\ref{sec:design:test}.

\assign{John}

\begin{itemize}

  \item{Examples.}

\end{itemize}

\subsection{CI system updates}
\label{sec:comp:ci}

Derived from \S\ref{sec:design:test}.

\assign{John}

\begin{itemize}

  \item{Test coverage.}
  \item{Tighter control of the environment.}
  \item{Better notifications.}
  \item{Better descriptions of which jobs do what.}
  \item{Clear description of what Developers are required to do before merging
  to master (see also \S\ref{sec:comp:doc}).}

\end{itemize}


\subsection{Metrics Dashboard / SQuaSH}

Derived from \S\ref{sec:design:test}.

\assign{Angelo}

Broadly as current SQuaSH, but to track:

\begin{itemize}

  \item{Code execution time.}
  \item{Test coverage?}
  \item{Notifications of regressions.}

\end{itemize}

\subsection{Standard format dataset package}
\label{sec:comp:dataset}

Derived from \S\ref{sec:design:test}.

\assign{Hsin-Fang}

The key considerations and motivations to standardize test datasets
in the construction phase include:

\begin{itemize}

\item{Developers would like low-friction access to test datasets.
A central location where developers can look up what has been curated
is desired.}

\item{Developers would like datasets at multiple scales and
representative of various data quality and observing conditions.
The properties of these datasets much be well understood.}

\item{Besides raw (unprocessed) data, intermediate and final data
products from various stages of pipeline processing are wanted.
They facilitiate testing of algorithms which are only relevant to
later parts of the pipeline or analysis codes without the need of
regenerating the products.}

\item{Datasets require continuing maintenance. Data products based
on a recent software release are usually wanted.}

\end{itemize}


The standard format of a dataset package is a ready-to-use Butler
repository and follows the format of a Butler repository as defined
in its corresponding obs package.  The format is configurable by
design, however, it is tied to the codes in the stack, so can change
from a software stack version to another.  Besides implementations
in the obs packages and Butler, other evolvement in the software
stack, such as handling of calibration data and reference catalog,
can also make a once-working repository incompatible.  Therefore,
maintenance is occassionally needed to ensure the usability of a
dataset package. \textbf{The QAWG recommends having a per-dataset product
owner.}\footnote{At the time of writing, our test datasets include
the following: afwdata, ap\_verify\_hits2015, testdata\_cfht,
testdata\_deblender, testdata\_decam, testdata\_jointcal,
testdata\_lsstSim, testdata\_subaru, qserv\_testdata,
validation\_data\_cfht.  validation\_data\_decam, validation\_data\_hsc,
/datasets/auxTel, /datasets/comCam, /datasets/ctio0m9, /datasets/lsstCam,
/datasets/decam /datasets/des\_sn, /datasets/hsc, /datasets/refcats,
/datasets/sdss, /datasets/gapon}
Product owners are responsible for ensuring that the content and
use cases of the datasets are well described and compatible with
recent stack versions.  The owner of a dataset can typically be the
team with immediate use cases and knowledge of its camera package.

The Obs Pkg WG (\jira{RFC-393}) is charged to re-design and refactor
the obs packages for maintainability and extensibility. We suggest
the Obs Pkg WG take into considerations in their design to mitigate
the close tie between a Butler repository and its obs package
implmentations, as well as adopt a common structure across different
cameras when possible.  After the refactoring, the obs packages
shall rarely change so the dataset format will be more stable.  The
QAWG recommends prioritise the Obs Pkg WG.

In some cases, a dataset pacakge may contain additional data that are not
tenable in the format of a Butler repository. We recommend following the
format as described in DM Developer Guide Common Dataset Organization
and Policy \footnote{\url{https://developer.lsst.io/services/datasets.html}}
and update the policy as needed.

As for the storage of test datasets, we consider any test dataset
package being either small or large, based on its size and use cases.
The QAWG's recommendations are as follows.

\begin{itemize}
  \item \textbf{Storage of small datasets}

  We consider small datasets as those smaller than around 100 GB
  and comfortably operatable as a Git LFS repository. They are
  carefully selected to meet specific use cases. The use cases of
  small datasets include

  \begin{itemize}
    \item{as input test data in CI jobs in the DM Jenkins system
          (\S\ref{sec:comp:ci});}
    \item{as example data in documentations, demos, and tutorials.}
  \end{itemize}

  To meet their needs, we recommend them

  \begin{itemize}
    \item{packaged as EUPS products;}
    \item{made publicly available on GitHub;}
    \item{stored as Git LFS repositories as needed;}
    \item{tagged their versions with the DM software releases;}
    \item{documented clearly its use cases and named product owner for each dataset;}
  \end{itemize}

  \item \textbf{Storage of large datasets}

  We consider large datasets as those larger than around 100 GB and
  hence transferring over network takes longer than an hour typically.
  They can contain edge cases that have not been identified to form
  specific small test datasets, or for use cases in which data
  volume is important.  We recommend them

  \begin{itemize}
    \item{made available on LSST development machines (currently on GPFS);}
    \item{usable and shared by team members;}
    \item{protected under a disaster recovery policy;}
    \item{documented clearly its use cases and named product owner for each dataset;}
  \end{itemize}

\end{itemize}



\subsection{Standard test package design}

Derived from \S\ref{sec:design:test}.

\assign{Hsin-Fang}

Currently, automatic continuous integration tests are performed via
multipla packages under two designs: (1) Scons-based execution, including
ci\_hsc and ci\_ctio0m9, and (2) exeuction through shell scripts
in validate\_drp.  Both ci\_hsc and validate\_drp are run in
Jenkins and triggered by timers every night (\S\ref{sec:comp:ci}).

It's QAWG's understanding that the validate\_drp scripts will
eventually replace ci\_hsc and ci\_ctio0m9, and a set of test scripts
will be run in a meta-package named lsst\_ci.  However, at the time
of writing, the validate\_drp scripts test only the single frame
processing step, while ci\_hsc exercises almost the entire end-to-end
DRP pipelines.  There has not been sufficient resources in implementing
further processing in validate\_drp.  Similarly, the lsst\_dm\_stack\_demo
repository should be converted into an EUPS product and a test
script added to lsst\_ci for execution (\jira{DM-14806}).

The QAWG recommends priority to unify the CI test package design and
finish the transition to validate\_drp. If such effort cannot
be allocated, documentations should be added to clearly describe
the status quo, and recommendations for developers during the
transition should be added to the Developer Guide (similar to
\S\ref{sec:comp:doc} and \S\ref{sec:comp:ci}). Before validate\_drp
can replace ci\_hsc and ci\_ctio0m9, the packages should be maintained.


Should address the union of lsst\_dm\_stack\_demo, ci\_hsc, validate\_drp use
cases.



\appendix
\glsaddall
\renewcommand*{\glsautoprefix}{glo:}
\printglossary[style=index,numberedsection=autolabel]

\bibliography{lsst,lsst-dm,refs_ads,refs,books}

\end{document}
