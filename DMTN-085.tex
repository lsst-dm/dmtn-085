\documentclass[DM,authoryear,toc,lsstdraft]{lsstdoc}
\usepackage[nonumberlist,nogroupskip,toc,numberedsection=autolabel]{glossaries}
\makeglossaries
%
% Acronyms
%

\newacronym{kpm}{KPM}{Key Performance Metric}
\newacronym{qawg}{QAWG}{QA Strategy Working Group}

%
% Terminology
%

\newglossaryentry{aggregation}
{
  name={aggregation},
  description={A single result---e.g., a \gls{metric value}---computed from a
  collection of input values. For example, we can sum or average a
  \gls{metric} computed over patches to produce an \gls{aggregate metric} at
  tract level}
}

\newglossaryentry{aggregate metric}
{
  name={aggregate metric},
  description={An \gls{aggregation} of multiple \glspl{point metric}. For
  example, the overall photometric repeatability for a particular tract given
  multiple observations of each star}
}

\newglossaryentry{dashboard}
{
  name={dashboard},
  description={A visual display of the most important information needed to
  achieve one or more objectives, consolidated and arranged on a single screen
  so that the information can be monitored at a glance \citep{Few:2013}}
}

\newglossaryentry{drill down}
{
  name={drill down},
  description={Move from a higher level aggregation of data to its inputs. For
  example, given data describing a tract, we might drill down to constituent
  patches and then to objects; given a visit, we might drill down to CCD and
  then source. In the context of this document, it refers to the act of
  identifying an issue in a high-level summary of the data (e.g. an aberrant
  \gls{metric value}) and interactively investigating its inputs to find the
  source of the problem}
}

\newglossaryentry{metric}
{
  name={metric},
  description={We follow the \citeds{SQR-019} definition of a metric as a
  measurable quantities which may be tracked. A metric has a name,
  description, unit, references, and tags (which are used for grouping). A
  metric is a scalar by definition. We consider multiple types of metric in
  this document; see \gls{aggregate metric}, \gls{model metric}, \gls{point
  metric}}
}

\newglossaryentry{metric value}
{
  name={metric value},
  description={The result of computing a particular \gls{metric} on some given
  data. Note that we \textit{compute}, rather than measure, metric values}
}

\newglossaryentry{model metric}
{
  name={model metric},
  description={A \gls{metric} describing a model related to the data. For
  example, the coefficients of a 2D polynomial fit to the background of a
  single CCD exposure}
}

\newglossaryentry{monitoring}
{
  name={monitoring},
  description={The process of collecting, storing, aggregating and visualizing
  metrics}
}

\newglossaryentry{point metric}
{
  name={point metric},
  description={A \gls{metric} that is associated with a single entry in a
  catalog. Examples include the shape of a source, the standard deviation of
  the flux of an object detected on a coadd, the flux of an source detected on
  a difference image}
}

\newglossaryentry{releaseable product}
{
  name={releaseable product},
  description={A software package or other component of the DM system which
  is expected to be included in the next tagged release of the system. At time
  of writing, this implies inclusion in a standard top-level package
  (e.g. lsst\_distrib), but we note that future changes to the release procedure
  may render that definition obsolete}
}

\newglossaryentry{tidy data}
{
  name={tidy data},
  description={Tidy datasets have a specific structure: each variable is a
  column, each observation is a row, and each type of observational unit is a
  table \citep{JSSv059i10}}
}


\input meta.tex

\title{QA Strategy Working Group Report}

\author{%
Bellm, E.C.,
Chiang, H.-F.,
Fausti, A.,
Krughoff, K.S.,
MacArthur, L.A.,
Morton, T.D.,
Swinbank, J.D and
Roby, T.
}

\setDocRef{\lsstDocType-\lsstDocNum}
\date{\vcsDate}

\setDocAbstract{%
Abstract.
}

\setDocChangeRecord{%
  \addtohist{\vcsRevision}{\vcsDate}{Unreleased draft.}{Bellm et al.}
}

% For specific WG recommendations
\newenvironment{recommendation}[1][]
{
  \begin{admonition}{green!5!white}{green!75!black}{Recommendation}{#1}
}
{
  \end{admonition}
}

\begin{document}

\maketitle

\section{Introduction}
\label{sec:intro}

This report constitutes the primary artefact produced by the DM \gls{qawg},
addressing its charge as defined in \citeds{LDM-622}.

\section{Approach to the Problem}

\subsection{Pipeline debugging}

\subsection{Deep drilling}

\subsection{Datasets and test infrastructure}

\section{Design sketch}

\subsection{Pipeline debugging}

\subsection{Deep drilling}

\subsection{Datasets and test infrastructure}

\section{Core components}

\subsection{Updated pipeline debugging system}

i.e. redesigned \texttt{lsstDebug}.

\subsection{Logging}

\subsection{Capability for developers to run pipelines at scale}

\subsection{Guidance on visualization}

We're requesting a set of guidelines for developers here, not a new framework
--- but that's still a concrete deliverable (it's just documentation, rather
than code). We might suggest that these guidelines be developed by a new WG,
per Simon's
suggestion\footnote{\url{https://confluence.lsstcorp.org/display/DM/Pipeline+Debugging+Design}}.

\subsection{Image viewer}

As of 2018-06-12 we haven't converged on a solid recommendation here.

Key considerations:

\begin{itemize}

  \item{Firefly is the annointed solution being provided by DM to external
  stakeholders (commissioning, operations, etc). It feels right to everybody
  that we should be dogfooding it, and also benefitting from development being
  carried out for those stakeholders.}

  \item{Currently, Firefly is unappealing to developers (primarily, I think,
  because of slowness of user interface, and perhaps also due to installation
  issues). Can we resolve these issues?}

  \item{We'd want to support visualization in a number of different
  environments, for e.g.:

    \begin{itemize}

      \item{Inside a Jupyter notebook;}
      \item{As a standalone tool, \`a la DS9;}
      \item{Embedded in a \gls{dashboard}, \`a la JS9, Aladin-Lite, etc.}

    \end{itemize}
  }

  \item{Do we lose flexibility by mandating the use of a backend-agnostic API
  (\texttt{afwDisplay}) rather than going ``all-in'' on e.g. a custom Firefly
  interface?}

  \item{We'll need to do full focal plane visualization, which none(?) of the
  current tools support well.}

\end{itemize}

Options include:

\begin{itemize}

  \item{Do nothing; continue as we are, which means most people will use DS9
  and a few will drift to Firefly as commissioning ramps up.}

  \item{Issue some sort of edict that pipelines developers have to use
  Firefly.}

  \item{Encourage the use of some other tool (Ginga?) instead of or as well as
  some of the above.}

  \item{Probably others.}

\end{itemize}

Sounds like we need somebody from the QAWG to actually write some requirements
--- or a wishlist set of features we want --- here.

\subsection{Catalog visualization tools}

For visualizing bigger-than-memory catalogs. May include e.g. the capability
to spin up Dask clusters on demand, combined with
Holoviews/Datashader/whatever. Somebody who knows about this stuff needs to
write a summary...

\subsection{Provenance}

This section should note:

\begin{itemize}

  \item{That provenance is an immediate issue impacting QA work, so a solution
  is a priority;}

  \item{Some requirements as to the granularity at which provenance tracking
  is necessary for QA.}

\end{itemize}

\appendix
\glsaddall
\printglossary[style=index]

\bibliography{lsst,lsst-dm,refs_ads,refs,books}

\end{document}
