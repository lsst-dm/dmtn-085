\documentclass[DM,authoryear,toc,lsstdraft]{lsstdoc}
\usepackage[nonumberlist,nogroupskip,toc,numberedsection=autolabel]{glossaries}
\makeglossaries
%
% Acronyms
%

\newacronym{kpm}{KPM}{Key Performance Metric}
\newacronym{qawg}{QAWG}{QA Strategy Working Group}

%
% Terminology
%

\newglossaryentry{aggregation}
{
  name={aggregation},
  description={A single result---e.g., a \gls{metric value}---computed from a
  collection of input values. For example, we can sum or average a
  \gls{metric} computed over patches to produce an \gls{aggregate metric} at
  tract level}
}

\newglossaryentry{aggregate metric}
{
  name={aggregate metric},
  description={An \gls{aggregation} of multiple \glspl{point metric}. For
  example, the overall photometric repeatability for a particular tract given
  multiple observations of each star}
}

\newglossaryentry{dashboard}
{
  name={dashboard},
  description={A visual display of the most important information needed to
  achieve one or more objectives, consolidated and arranged on a single screen
  so that the information can be monitored at a glance \citep{Few:2013}}
}

\newglossaryentry{drill down}
{
  name={drill down},
  description={Move from a higher level aggregation of data to its inputs. For
  example, given data describing a tract, we might drill down to constituent
  patches and then to objects; given a visit, we might drill down to CCD and
  then source. In the context of this document, it refers to the act of
  identifying an issue in a high-level summary of the data (e.g. an aberrant
  \gls{metric value}) and interactively investigating its inputs to find the
  source of the problem}
}

\newglossaryentry{metric}
{
  name={metric},
  description={We follow the \citeds{SQR-019} definition of a metric as a
  measurable quantities which may be tracked. A metric has a name,
  description, unit, references, and tags (which are used for grouping). A
  metric is a scalar by definition. We consider multiple types of metric in
  this document; see \gls{aggregate metric}, \gls{model metric}, \gls{point
  metric}}
}

\newglossaryentry{metric value}
{
  name={metric value},
  description={The result of computing a particular \gls{metric} on some given
  data. Note that we \textit{compute}, rather than measure, metric values}
}

\newglossaryentry{model metric}
{
  name={model metric},
  description={A \gls{metric} describing a model related to the data. For
  example, the coefficients of a 2D polynomial fit to the background of a
  single CCD exposure}
}

\newglossaryentry{monitoring}
{
  name={monitoring},
  description={The process of collecting, storing, aggregating and visualizing
  metrics}
}

\newglossaryentry{point metric}
{
  name={point metric},
  description={A \gls{metric} that is associated with a single entry in a
  catalog. Examples include the shape of a source, the standard deviation of
  the flux of an object detected on a coadd, the flux of an source detected on
  a difference image}
}

\newglossaryentry{releaseable product}
{
  name={releaseable product},
  description={A software package or other component of the DM system which
  is expected to be included in the next tagged release of the system. At time
  of writing, this implies inclusion in a standard top-level package
  (e.g. lsst\_distrib), but we note that future changes to the release procedure
  may render that definition obsolete}
}

\newglossaryentry{tidy data}
{
  name={tidy data},
  description={Tidy datasets have a specific structure: each variable is a
  column, each observation is a row, and each type of observational unit is a
  table \citep{JSSv059i10}}
}


\input meta.tex

\title{QA Strategy Working Group Report}

\author{%
Bellm, E.C.,
Chiang, H.-F.,
Fausti, A.,
Krughoff, K.S.,
MacArthur, L.A.,
Morton, T.D.,
Swinbank, J.D and
Roby, T.
}

\setDocRef{\lsstDocType-\lsstDocNum}
\date{\vcsDate}

\setDocAbstract{%
Abstract.
}

\setDocChangeRecord{%
  \addtohist{\vcsRevision}{\vcsDate}{Unreleased draft.}{Bellm et al.}
}

% For specific WG recommendations
\newenvironment{recommendation}[1][]
{
  \begin{admonition}{green!5!white}{green!75!black}{Recommendation}{#1}
}
{
  \end{admonition}
}

\begin{document}

% For specific WG recommendations
\newcommand{\assign}[1]{
  \begin{admonition}{red!5!white}{red!75!black}{Assignee}{}#1\end{admonition}
}

\maketitle

\section{Introduction}
\label{sec:intro}

\assign{John}

This report constitutes the primary artefact produced by the DM \gls{qawg},
addressing its charge as defined in \citeds{LDM-622}.

\section{Approach to the Problem}

\assign{John}

\subsection{Pipeline debugging}

\assign{John}

What tools do we need to help pipeline developers with their everyday work?

\begin{itemize}

  \item{How do you go about debugging a \texttt{Task} that is crashing?}
  \item{Is \texttt{lsstDebug} adequate?}
  \item{Do we need an afwFigure, for generating plots, to go alongside \texttt{afwDisplay}, for showing images?}
  \item{What additional capabilities are needed for developers running and debugging at scale, e.g. log collection, identification of failed jobs, etc.}
  \item{What's needed from an image viewer for pipeline developers? Is DS9 or Firefly adequate? Is there value to the afwDisplay abstraction layer, or does it simply make it harder for us to use Firefly's advanced features?}
  \item{How do we view images which don't fit in memory on a single node?}
  \item{How do we handle fake sources? Is this a provenance issue?}

\end{itemize}

\subsection{Drill down}

\assign{John}

\subsection{Datasets and test infrastructure}

\assign{John}

\section{Design sketch}

\subsection{Pipeline debugging}

\assign{John}

\subsection{Drill down}

\assign{Tim}

\subsection{Datasets and test infrastructure}

\assign{John}

\section{Core components}

\subsection{Updated pipeline debugging system}

\assign{Simon}

i.e. redesigned \texttt{lsstDebug}.

\subsection{Logging}

\assign{Simon}

\subsection{Capability for developers to run pipelines at scale}

\assign{Lauren}

\subsection{Guidance on visualization}

\assign{Lauren}

We're requesting a set of guidelines for developers here, not a new framework
--- but that's still a concrete deliverable (it's just documentation, rather
than code). We might suggest that these guidelines be developed by a new WG,
per Simon's
suggestion\footnote{\url{https://confluence.lsstcorp.org/display/DM/Pipeline+Debugging+Design}}.

\subsection{Image viewer}

\assign{Trey}

There are two approaches to doing image visualization.  When a developer is working on a laptop with a
small amount of data and viewing image results locally then DS9 or an equivilent desktop application seems
to be the best solution.

In the case where the data is on some remote server then visualization needs to
be more advanced. Here we recommend using Firefly web based image visulization. A large Firefly server(s) near the
data will allow quicker and simpler visualization. The data is not required to be moved to the laptop to visualize.

To be effective for QA Firefly should have the following functionallity:
\begin{itemize}
  \item{Work as a standalone tool. }
  \item{Work with the afw.display API.}
  \item{Provide an API that allows a user to acess more advanced features than afw.display provides.
  Possibly afw.display API should be enhanced in certain cases.}
  \item{Work as an extension in Jupyter Lab. For example- click on a FITS file and see it show in Firefly.}
  \item{Work as a Jupyter Lab widget.}
  \item{Provide a way to show tabular data on images.}
  \item{Provide HiPS viewing.}
  \item{Work with all major QA use cases and ensure good performance.}
  \item{Implement full focal plane visualization.}
  \item
\end{itemize}

What else do we want here?


\subsection{Catalog visualization tools}

\assign{Lauren}

For visualizing bigger-than-memory catalogs. May include e.g. the capability
to spin up Dask clusters on demand, combined with
Holoviews/Datashader/whatever. Somebody who knows about this stuff needs to
write a summary...

\subsection{Provenance}

\assign{Hsin-Fang}

This section should note:

\begin{itemize}

  \item{That provenance is an immediate issue impacting QA work, so a solution
  is a priority;}

  \item{Some requirements as to the granularity at which provenance tracking
  is necessary for QA.}

\end{itemize}

\appendix
\glsaddall
\printglossary[style=index]

\bibliography{lsst,lsst-dm,refs_ads,refs,books}

\end{document}
